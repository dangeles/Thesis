%%%%%%%%%%%%
%% Please rename this main.tex file and the output PDF to
%% [lastname_firstname_graduationyear]
%% before submission.
%%%%%%%%%%%%

\documentclass[12pt]{caltech_thesis}
\usepackage[hyphens]{url}
\usepackage{lipsum}
\usepackage{graphicx}
\graphicspath{{images/}}
\usepackage{todonotes}
\usepackage{adjustbox}
\usepackage{csvsimple} %insert tables using .csv files


% activates the subbottom and subtop environments in memoir
% \newsubfloat{figure}

%% Tentative: newtx for better-looking Times
\usepackage[utf8]{inputenc}
\usepackage[T1]{fontenc}
\usepackage{newtxtext,newtxmath}
\usepackage{nameref}
% line numbers
\usepackage[right]{lineno}
% improves typesetting in LaTeX
% \usepackage{microtype}
% degree symbol
\usepackage{gensymb}
\usepackage{amsmath}
\usepackage{booktabs}
\usepackage{color}
\definecolor{Gray}{gray}{.25}

\usepackage[colorlinks = true,
            linkcolor = blue,
            urlcolor  = blue,
            citecolor = blue,
            anchorcolor = blue]{hyperref}

%commands
\newcommand{\qval}[1]{$q<10^{-#1}$}
\newcommand{\cel}{\emph{C.~elegans}}
\newcommand{\ecol}{\emph{E.~coli}}

% % gene names
\newcommand{\gene}[1]{\mbox{\emph{#1}}}
\newcommand{\genotype}[1]{\mbox{\emph{#1}}}
\newcommand{\protein}[1]{\mbox{\uppercase{#1}}}

% Must use biblatex to produce the Published Contents and Contributions,
% per-chapter bibliography (if desired), etc.
\usepackage[
    backend=biber,natbib,
    % IMPORTANT: load a style suitable for your discipline
    style=authoryear
]{biblatex}

% Name of your .bib file(s)
\addbibresource{hypoxia.bib}
\addbibresource{female-transcriptome.bib}
\addbibresource{allelic_series.bib}
\addbibresource{tea.bib}
\addbibresource{pea.bib}
\addbibresource{ownpubs.bib}

\usepackage{siunitx}
\usepackage[algo2e]{algorithm2e}

\DoubleSpacing{}

\begin{document}

% Do remember to remove the square bracket!
\title{A theory of genetic analysis using transcriptomic phenotypes}
\author{David Angeles-Albores}

\degreeaward{Doctor of Philosophy}                 % Degree to be awarded
\university{California Institute of Technology}    % Institution name
\address{Pasadena, California}                     % Institution address
\unilogo{caltech.png}                                 % Institution logo
\copyyear{[2018]}  % Year (of graduation) on diploma
\defenddate{[18 September 2018]}          % Date of defense

\orcid{0000--0001--5497--8264}

%% IMPORTANT: Select ONE of the rights statement below.
\rightsstatement{All rights reserved}

%%  If you'd like to remove the Caltech logo from your title page,
% simply remove the "[logo]" text from the maketitle command
\maketitle[logo]

\linenumbers{}

\begin{acknowledgements}
  This thesis is possible thanks to the unwavering support of a long list of
  individuals. I would like to thank my advisor, Paul W. Sternberg, for
  providing a laboratory and an intellectual home for the past few years. I will
  always remember our conversations over coffee at 9am in the lab. I taught
  Systems Genetics with Paul as my project came into full maturity; these
  lectures on genetics gave us a chance to recast the classical interpretations
  with a genomics perspective. Those lectures are some of my fondest memories
  at Caltech. I also need to acknowledge my thesis committee, Dianne K. Newman,
  Elliot Meyerowitz and Matt Thomson. Without their advice, I would be
  considerably more confused than I am today. Dianne has been a major figure
  throughout my Ph.D., a great scientist with a heart to match, and I feel
  lucky to have had an opportunity to learn from her insights. I also need to
  thank Erich M. Schwarz, who taught me to argue, and taught me to write. Where
  others were content to say my work was fine, Erich found every possible
  loophole, every minor detail and every open question and pushed me to be
  complete without being redundant. I am sincerely grateful for his guidance and
  his mentorship.

  I have been very lucky to have benefitted from a fantastic set of
  collaborators in the Sternberg lab. I have been lucky to work with Hillel
  Schwartz, a fantastic geneticist and good friend; Carmie Puckett Robinson,
  with whom I started to work on transcriptome genetics; Daniel Leighton, who
  taught me about worm pheromones and aging; Raymond Y. Lee and Juancarlos Chan,
  with whom I learned all the intricacies of WormBase and tool design.
  Throughout my time here, I have worked with three extremely talented
  undergraduates who made working in the lab much more exciting: Tiffany Tsou,
  Kyung Hoi (Joseph) Min and Vladimir Molchanov. Finally, I need to thank all
  the members of the Sternberg lab for making science come to life: Jon Liu, Han
  Wang, James Lee, Pei-Yin, Katie Brugman, Cynthia Chai, Wen Chen, Sarah Cohen,
  Elizabeth Holman, Sandy Wong, Heenam Park, Daniel Jun Oh, Ravi Nath, Margaret
  Ho, Srimoyee Ghosh, Sophie Walton, Sarah Torres, Shahla Gharib, Barbara Perry,
  Animesh Ray and Elizabeth Glater. I cannot name all of the friends I have
  made here at Caltech; I hope they know how grateful I am for their friendship.

  I would like to briefly acknowledge the programs and the people who brought me
  to Caltech. I would not be here without the EXtraordinary Research
  Opportunities Program (EXROP) from HHMI, where I met Andrew Quon and Christy
  Schultz. Through EXROP I met and had a chance to work for Susan Lindquist and
  her (then) postdoc Georgios Karras, who taught me the beauty of yeast
  genetics. I wish I could show Sue what I have done with the doors that she
  opened for me. At Cornell, I was incredibly lucky to be adviced by Laurel
  Southard, who believed in my potential no matter what grades might say.

  Throughout my time at Caltech, I have never been alone. My family has been a
  source of unconditional support. I thank my parents, Lilia and Josué, and my
  brother, Andrés, for always believing in me.

  Finally, I would like to give my heartfelt thanks to Heather Curtis.

  Heather, you have been the sun, and the moon, and the stars in my life since I
  met you. You brought new colors into my world. Every day, I learn to think in
  new ways, I learn to see new things, thanks to you. I am a better person
  because I am with you and I am grateful that life brought us together.

  \emph{This thesis is for you.}
\end{acknowledgements}

\begin{abstract}
  This thesis deals with the conceptual and computational framework required to
  use transcriptomes as effective phenotypes for genetic analysis. I demonstrate
  that there are powerful theoretical reasons why Batesonian epistasis should
  feature prominently in transcriptional phenotypes. I also show how to compute
  and interpret the aggregate statistics for transcriptome-wide epistasis and
  transcriptome-wide dominance using whole-organism transcriptomic profiles of
  \cel{} mutants. Finally, I developed the WormBase Enrichment Suite for
  enrichment analysis of genomic data.
\end{abstract}

%% Uncomment the `iknowhattodo' option to dismiss the instruction in the PDF.
\begin{publishedcontent}[iknowwhattodo]

% List your publications and contributions here.
\nocite{Angeles-Albores2018a, Angeles-Albores2018b,
        Angeles-Albores2017, Angeles-Albores2016}
\nocite{Angeles2018micro}
\end{publishedcontent}

\tableofcontents
\listoffigures
\listoftables

\mainmatter{}

\chapter*{Preface}
\addcontentsline{toc}{chapter}{Preface}

I have tried to organize this thesis in a way that makes sense. Briefly, the
thesis can be viewed in three segments: Epistasis, complementation and software
development. In doing so, I have broken the chronological order of my work, but
I do not see this as a problem. Science is rarely linear, and it may often turn
out that the last concepts to be found are actually those concepts that allow us
to make sense of everything else. This has certainly been the case with my work.

In Chapter 1, the reader will find a brief overview of the problem facing
transcriptome genetics. This chapter encompasses a review of the relevant
literature, but beyond that, I have tried to make arguments I think are
important. First, transcriptome genetics has obviated the chasm between
statistical epistasis and classical, or Batesonian, epistasis. The confusion
between the two (related) terms has been one of the great misfortunes in the
field of genetics, since it has hampered a significant amount of work. I am glad
to say that in this thesis I have achieved the unification of both concepts,
such that no confusion should happen. Second, although we now know how to search
signs of epistasis and dominance in transcriptomes, the issue of counting
phenotypic classes or modules is becoming increasingly ominous. Unless and until
we can confidently identify and purge spurious modules, we will not be able to
use these phenotypes to their full extent.

In Chapter 2, I have written a theoretical argument that is the basis for the
rest of the chapters dealing with epistasis. In this chapter, I prove that
epistasis emerges from statistical mechanics, such that even genes that have
enormously complex transcriptional mechanisms can in some cases exhibit
Batesonian epistasis. This chapter establishes genetics as a variational method
with which to probe an unknown partition function, enabling us to make statements
about what values the partition function is or is not allowed to take.

In Chapter 3, I develop the concept of transcriptome-wide epistasis and use
it to reconstruct the well-studied hypoxia pathway. In Chapter 4, I use the
concept of transcriptome-wide epistasis to identify a novel stage in the life
cycle of the roundworm \cel{}.

Chapter 5 deals with the issue of complementation, and its study through
expression profiles. In my opinion, this is the most complicated chapter in this
thesis. I struggled with every aspect of this project, but the result is, to my
mind, pleasing.

Chapter 6 and 7 deal with the creation of the WormBase Enrichment Suite.

Throughout this thesis, I have tried to be pedagogical. If we don't make efforts
to explain the computational methods we are developing, biology will pass from
a scientific discipline to an astrological pseudo-science, and we will fail to
see the true beauty in the stars above and instead imbue them with our human
desires and flaws, asking them to help us reach fame instead of helping us to
solve the mysteries that abound in our universe.


% for some reason the introduction.bib file isn't working
% so all the references are in the hypoxia.bib file for this chapter... sigh
\chapter{Introduction}
\begin{refsection}
  \section*{Prologue}

\section*{A short history of genetic analysis}

\subsection*{Epistasis and genetic interactions}

\subsection*{Birth of genomics}

\subsection*{Genetical genomics}

\section*{Overview of the problem}

  \printbibliography[heading=subbibliography]
\end{refsection}

\chapter{A Statistical Mechanical Theory of Genetics using Gene Expression
         Phenotypes}
\begin{refsection}
  
\newcommand{\Freg}{\( F_{reg} \)}


\section*{Abstract}
\textbf{Genetics has been a cornerstone of biology since its inception almost
100 years ago. However, the rules of genetic inference remain
axiomatic, and as a result, confusion regarding these rules abounds. Here, I
show that Batesonian epistasis, which has been used to construct genetic
networks, emerges naturally from a statistical mechanical framework. I will show
how the inferences of genetic interactions on the basis of Batesonian epistasis
emerge from this framework, thus establishing that the methods geneticists use
to identify interactions amongst genes can be viewed as a non-analytical
variational method to probe an unknown partition function. By establishing
Batesonian epistasis, this method can constrain the properties of this unknown
partition function and reveal how variables within this partition function are
functionally related to each other.
}

\section*{Introduction}
Biological processes can be broadly subdivided into two large categories:
Processes where the players and interactions are known, and processes where the
players or the interactions are unknown. Necessarily, the study of processes in
either category varies substantially, as do the available methodologies.
Processes where the components are unknown cannot be studied in a predictive
fashion and mathematical methods cannot be used to identify the components that
participate in the process (though they can make statements about the properties
these components must exhibit). On the other hand, when the major components and
interactions of a system are known, detailed models can be built to study
emergent properties present in the system or to identify the regimes in which
the model breaks down, thus indicating missing information about the system.

A particularly effective method for modeling a large number of physical
phenomena is statistical mechanics. Statistical mechanics dictates that if all
the possible configurations of a system are known (all the states and their
weights), then the probability that an event happens is equal to the weighted
sum of the states that can lead to that event divided by the weighted sum of all
the possible states. The weighted sum of all the possible states is particularly
important and is called the Partition Function. Though this theory was
originally applied to ideal gases, it has become widespread, finding
applications in all areas of physics, but also in biology~\citep{Garcia2007},
neurobiology~\citep{Schneidman2006} and even the social
sciences~\citep{Lee2015}. Statistical mechanical models demand a detailed and
quantitative knowledge of the system to be tractable.

In contrast to the quantitative description of a physical phenomenon obtained
from statistical mechanical models, genetics is a favored approach when the
system is completely unknown, because the genetical framework is explicitly
designed to make as few assumptions as possible. Saturating genetic screens can
be used to find all of the genes that cause a specific phenotype when
mutated~\citep{Brenner1974,Jurgens1984}. Once these genes are known, epistatic
analyses using null mutants of these genes can establish genetic interactions
between these genes.

Genetic interactions describe the broadest possible mode of interaction in a
biological system. Formally, two genes genetically interact if they affect the
same phenotype. However, not all gene interactions are considered equally
informative; the set of uninformative genetic interactions are defined by the
choice of null hypothesis, which is often selected to be a multiplicative null
hypothesis (also known as log-additive null hypothesis). A more strict and
informative definition of genetic interactions requires:

\begin{enumerate}
  \item The two genes control the same phenotype;
  \item The phenotype of the double null mutant deviates from that expected
  under a null model of interaction
\end{enumerate}

The deviation between the expected phenotype and the observed phenotype of the
double mutant is an important quantity that we refer to as generalized
epistasis~\citep{Fisher1919}. Although generalized epistasis can take on any
value in theory, in the context of developmental genetics, generalized epistasis
measured between two genes using null mutants often reduces to a limited set of
specific values. These values correspond to a phenomenon we refer to as complete
epistasis, which is important for classical developmental genetics. Complete
epistasis refers to a situation where the double mutant exhibits exactly the
same phenotype as one of the single mutants, originally defined by
Bateson~\citep{Bateson2009}. Measuring this equality is equivalent to
constraining the value of the generalized epistasis term to a unique value.
The relationship between classical and generalized epistasis has been a source
of great confusion, not least because multiple fields use conflicting
definitions for the term and use it to measure different
things~\citep{Cordell2002,Phillips2008}. A major problem is that the presence of
generalized epistasis depends strongly on the choice of null model. In this
text, we only use the term epistasis, either complete or generalized, to refer
to the situation where the double mutant exhibits the same phenotype as one of
the corresponding single mutants. We will use the word `mutant' to refer to a
null allele, unless otherwise specified. Finally, we will use the term
`epistasis coefficient' to refer to the statistical quantity of generalized
epistasis, which depends on a null model.

Although genetics is used to study physical systems, there is not a systematic
mapping of the language of genetics into the language of statistical mechanics.
A consequence of this is that genetic interactions are often considered vague to
the point of being uninformative. Here, we show that genetics is a method for
studying properties of an unknown partition function. Identifying genetic
interactions between two components helps constrain the functional form of the
partition function, which in turn constrains the states that are accessible to
the system.

\section*{Statistical Mechanics of Complete Epistasis}
We have previously observed complete epistasis using gene expression profiles as
phenotypes~\citep{Angeles-Albores2017,Angeles-Albores2018a}. To study how this
can occur, we use a model of gene expression derived from statistical mechanics
that has been used to accurately describe various promoter architectures in
multiple organisms~\citep{Bintu2005a,Garcia2007,Bothma2015,Raveh-Sadka2009}.
Briefly, the model can be expressed in the following functional form (for a
detailed and friendly derivation, see \citet{Bintu2005a} or \citet{Garcia2007}):

\begin{equation}
  p_{bound}(A, B, \ldots) = \frac{1}{1 + \frac{1}{\rho F_{reg}(A, B, \ldots)}}.
  \label{eq:pbound}
\end{equation}

Here, \(p_{bound}(A, B, \ldots)\) is the probability that RNA Polymerase is
bound to the locus of interest (this probability is in turn proportional to the
expression level of this gene~\citep{}), and which depends on several factors
\(\rho, A, B, \ldots \); \(\rho \) describes the concentration of RNA polymerase
bound to the promoter; \Freg{} is a factor that modulates the effective number
of RNA polymerases at the promoter of interest. This factor can be used to model
a variety of transcription factors, such as activators, or inhibitors, and can
be used to model interactions between these factors. Its exact value will depend
on factors \(A, B, \ldots \). These factors represent the activities of the gene
products of genes \gene{a}, \gene{b}, \dots. For simplicity, capital letters
will always represent compound activities of the final gene products (proteins),
whereas lower-case italicized letters will always represent the genes encoding
said products, in accord with \cel{} genetic nomenclature rules.
Equation~\ref{eq:pbound}, although superficially simple, is able to accomodate
an enormous variety of promoter architectures through its dependence on \Freg{}.

For example, consider a system where two proteins, \(A\) and \(B\) can independently
bind different sequences on a promoter, and can independently bind RNA
Polymerase. Suppose that \(A\) and \(B\) do not interact with each other. Then,
\Freg{} takes the following functional form:

\begin{equation}
    F_{reg}(A, B) = \frac{1 + A e^{-\varepsilon _{AP}} + B e^{-\varepsilon _{BP}} }
                         {1 + A + B},
\end{equation}
where \(\varepsilon _{XP}\) is the interaction energy between protein \(X\) and RNA
Polymerase. If \(A\) and \(B\) can bind to each other as well as to RNA
polymerase, then \Freg{} becomes:

\begin{equation}
    F_{reg}(A, B) = \frac{\frac{1 + Be^{-\varepsilon _{BP}}}{1 + B} +
                          A e^{-\varepsilon _{AP}} \cdot
                              \frac{1 + B e^{-\varepsilon _{AB}-\varepsilon _{BP}}}
                                   {1 + B}
                          }
                         {1 + A\frac{1 + Be^{-\varepsilon_{AB} } }
                                    {1 + B}
                         },
\end{equation}
where \(\varepsilon_{XY}\) denotes the energy of interaction between protein \(X\)
and protein \(Y\). Finally, in the case where protein \(B\) binds DNA but does
not interact with RNA polymerase (\(\varepsilon_{BP} = 0\)), this equation
simplifies to:
\begin{equation}
    F_{reg}(A, B) = \frac{1 +
                          A e^{-\varepsilon _{AP}} \cdot
                              \frac{1 + B e^{-\varepsilon _{AB}}}
                                   {1 + B}
                          }
                         {1 + A\frac{1 + Be^{-\varepsilon_{AB} } }
                                    {1 + B}
                         },
\end{equation}
We have shown how \Freg{} accomodates a variety of activator architectures. \Freg{}
can similarly accomodate a variety of repressive architectures as well as mixed
cases where proteins are both repressors and activators.

Next, we suppose that the two genetic factors we are studying can be expressed
via \Freg. Using our transcriptional reporter as a phenotype, we would like
to know what the conditions are for complete epistasis to occur. Let the
two genes under study be \gene{a} and \gene{b}, with protein products \(A\) and
\(B\) respectively. Since epistasis analyses can only be safely carried out using
null alleles, we can specify epistasis in our toy system as an equation (we let
\gene{a} be the epistatic factor without loss of generality)\footnote{For
simplicity, we represent the effect of a null mutation as deleting all
protein product. Another way to generate null alleles is by eliminating all
biochemical activities, which would correspond to setting all the interaction
energies for a given protein to zero. This alternative representation does not
affect our argument.}:

\begin{equation}
  p_{bound}(0, 0) = p_{bound}(0, B).
  \label{eq:epistasis}
\end{equation}

Since \(p_{bound}\) depends on \(A\) and \(B\) only through the factor,
\(F_{reg}(A, B)\), Eq.~\ref{eq:epistasis} can be re-written to more explicitly
show the requirement for epistasis in terms of the regulatory function,
\Freg{}:

\begin{equation}
    F_{reg}(0, 0) = F_{reg}(0, B).
    \label{eq:epistasis2}
\end{equation}

We are interested in identifying a mathematical condition that will satisfy
Eq.~\ref{eq:epistasis} and Eq.~\ref{eq:epistasis2} and which is biologically
relevant. We noticed that Eq.~\ref{eq:epistasis2} can be guaranteed if the
variables \(A\) and \(B\) can be combined into a single compound variable:

\begin{equation}
    F_{reg}(A, B) = F_{reg}[A\cdot{G_{reg}(B)}].
\end{equation}

In other words, Eq.~\ref{eq:epistasis} is satisfied if the only function of
\gene{b} is to genetically alter the effective genetic activity of \gene{a}
through a regulatory function, \(G_{reg}\). This condition is relatively lax, and
in fact is frequently a property of signaling pathways\citep{} and of many
promoter architectures~\citep{Bintu2005a}. No conditions are imposed on how
\gene{a} interacts with the promoter.

\subsection*{Genetic suppression does not require the two factors interact
             physically on the promoter sequence}
In the above section, we modeled a situation where the protein products of
our genes interacted directly with the promoter that drives the transcriptional
reporter. However, the situation does not change at all if the protein products
never interact with the promoter, but instead drive activity of another factor,
\(alpha\), that is the physical interactor. The numeric results may change, but
the equality holds, as long as \(alpha\) accepts the inputs \(A\) and \(B\)
in compound form. In other words, if \(\alpha(A, B)\) can be written in the form
\(\alpha[h(A)\cdot j(B)]\), then the equality in Eq.~\ref{eq:epistasis} holds.
In other words, complete epistasis can occur between factors that are physically
and/or temporally separated. Conversely, if two factors show complete epistasis,
the only constraint this imposes on their interactions is that one gene must
alter the effective gene activity of the other, and that the hypostatic gene
(the gene for which the phenotype is masked) not interact with the phenotype
through another pathway that is independent of the epistatic gene. Thus, genetic
interaction can abbreviate the pathway components between the interacting genes,
\gene{a} and \gene{b}, and represent them through single arrows. The above
arguments strongly suggest that the genetic `distance' between the genes under
study and the transcriptional phenotype is irrelevant.

\begin{figure}
  \includegraphics[width=\linewidth]{theoryims/Branching_Abbreviation.pdf}
  \caption{Signaling pathways can be abbreviated in genetic pathway
           representations.
           \textbf{A}. Two genes, \gene{a} and \gene{b},
           encoding protein products A and B respectively, interact to drive
           expression of a green gene, which in turn drives expression of two
           blue genes (through potentially distinct mechanisms). The blue genes
           then interact to drive expression of an orange gene. \gene{a} is
           epistatic over \gene{b}. Removing \gene{a} sends the green node to a
           a specific expression state. This state drives the blue genes into
           a specific state as well; then the blue genes drive the orange gene
           into a well-defined state. Removing \gene{b} sends the network into a
           completely different state. Since \gene{a} is epistatic over
           \gene{b}, mutating both genes sends the network into the same state
           as mutating \gene{a} alone.
           \textbf{B}. Since the network can only exist in three states (a
           wild type state, an \(a^-\) state and a \(b^-\) state), the network
           can be abbreviated to reflect that the state of the orange node can
           be completely specified by knowing the state (present or missing) of
           the upstream factors, \gene{a} and \gene{b}. \textbf{C}. This
           corresponds to stating that the probability that the orange node is
           transcribed can be expressed solely in terms of the protein products
           A and B when the products are expressed at wild-type levels or when
           one or the other product is completely missing.
           }\label{fig:branch}
\end{figure}

Another important question is whether there is an `optimal' choice of expression
phenotype. In other words, for a given graph containing nodes (genes) that can
interact (arrows) with other nodes, is there a `best' node to select as a
readout of the pathway (see Fig.~\ref{fig:branch})? For simplicity, but without
loss of generality, we envision a graph where a single, primary node accepts as
an input our two factors \(A\) and \(B\). This node can in turn drive expression of
a second layer of genes through arbitrary activation functions, and this second
layer can drive expression of a third layer of (completely different) genes.
Genes in this second layer are allowed to have any arbitrary interaction with
other genes in the second layer to drive expression of genes in the third layer.
This layering can repeat arbitrarily. We seek to answer the question: For an
arbitrary, acyclical graph, is there an `optimal' node that will reveal the
epistatic interactions between the two genetic factors \gene{a} and \gene{b},
assuming perfect measurements and zero biological noise?

Let us represent the expression level of the \(j\)th node in the \(i\)th layer
as \(p_{b, ij}\) following our previous notation. Clearly, the expression of the
primary node, \(p_{b, 00}\) depends on \(A\) and \(B\). We can write the
dependence of the state on these factors explicitly: \(p_{b, 00}(A, B)\). Then,
it follows that loss of \(A\) or \(B\) will send the state of this node to the
specific states, \(p_{b, 00}(0, B)\) and \(p_{b, 00}(A, 0)\) respectively. Since
\gene{a} is epistatic over \gene{b}, and \(A\) and \(B\) interact to drive this
node directly, then it follows that \(p_{b, 00}(0, B) = p_{b, 00}(0, 0)\). The
value \(p_{b, 00}(0, B)\) is an important quantity, therefore we assign it the
name \(P_{00}\). This proves node 00 can function as a suitable phenotype for
genetic analysis (by definition).

The second set of nodes \(p_{b, 1j}\) are regulated directly by node 00. We take
the activity of node 00 to be directly proportional to \(p_{b, 00}(A, B)\). Thus,
for any node 1\(j\), we can represent its expression level as
\(p_{b, 1j}(T[p_{b, 00}(A, B)])\), where \(T[\cdot]\) represents an arbitrary
time-independent function that transforms the expression level of 00,
\(p_{b, 00}\), into its product activity. Deletion of \gene{a} or \gene{b}
transmits directly into the expression states \(p_{b, 1j}(T[P_{00}])\) and
\(p_{b, 1j}(T[p_{b, 00}(A, 0)])\) respectively. Deleting \gene{a} and \gene{b}
simultaneously will cause the any node in layer 1 to enter the state
\(p_{b, 1j}(T[P_{00}])\). Since \(T[\cdot]\) is a time-independent function, it
follows that any node in layer 1 can be used as a gene expression phenotype,
regardless of any well-defined, non-trivial dependence it may have on node 00.
We can iterate this logic throughout any arbitrary number of layers. A layer can
drive expression of the next layer in any way without altering the ability of
the next set of nodes to function as indicators of the epistasis relationship
between \gene{a} and \gene{b}.

Taken together, these results explain why genetic diagrams need not represent
all factors between a pathway, and why genetic pathways are broadly insensitive
to spatial or temporal separation between components: Biological and
experimental noise would be the only limitations to detecting epistasis deep
into a directed acyclic graph. These results also highlight the fact that the
\emph{genetic diagrams} that emerge from time-independent epistatic analysis are
not necessarily informative about the dynamical behaviors of the \emph{molecular
interactions} that they result from because our genetic diagrams are constructed
in a digital fashion (presence or absence of a factor) and the phenotypes do not
contain temporal information that could inform us about these dynamics.

% \section*{The epistasis coefficient}
% \citet{Angeles-Albores2018a} introduced the notion of an epistasis coefficient.
% Briefly, the epistasis coefficient for gene expression measurements is defined
% as:
%
% \begin{equation}
%     s(a, b) = \frac{\log{FC_{ab} - \log{FC_a} - \log{FC_b}}}
%                     { \log{FC_a} + \log{FC_b}}.
% \label{eq:coeff}
% \end{equation}
% Here, $s(a, b)\) is the epistasis coefficient between \gene{a} and \gene{b};
% $FC_{x}\) is the fold-change in expression in mutants of genotype $x$ with
% respect to the wild-type; and $\log$ refers to the natural logarithm.
% Equation~\ref{eq:coeff} takes this form because we specified the null hypothesis
% of interaction to be log-additive. As a result, we can rewrite this equation
% in terms of observations and expectations:
%
% \begin{equation}
%     s(a, b) = \frac{\mathrm{Observed} - \mathrm{Expected}}{\mathrm{Expected}}.
% \end{equation}
%
% The epistasis coefficient can also be re-expressed using the language of
% statistical mechanics, if we define the fold-change to be the ratio of
% \(p_{bound}\) in the mutant relative to \(p_{bound}\) in the wild-type:
% $FC_x = p_{bound}^{mt}/p_{bound}^{wt}\). Performing
% this substitution allows us to inspect the values that $s(a, b)\) can take if \(A\)
% is epistatic to \(B\); that is to say, if Eq.~\ref{eq:epistasis} holds. Since
% Eq.~\ref{eq:epistasis}  is an identity, we can substitute $\log{FC_{ab}}\) with
% $\log{FC_{a}}\) in Eq.~\ref{eq:coeff}. Thus, when \gene{a} is epistatic to
% \gene{b}, Eq.~\ref{eq:coeff} simplifies to:
%
% \begin{equation}
%     s(a, b) = -\frac{\log{FC_b}}{\log{FC_a} + \log{FC_b}}.
% \end{equation}
%
% With this simplified form, we can ask what will happen as $FC_a$ changes.
% Consider the case where $FC_a = 1$, which means the epistasis coefficient is -1.
% This reflects the case where \gene{b} is a strong inhibitor of the gene activity
% of \gene{a}, such that $\gene{a}\) is not active in the experimental conditions
% (removing \gene{a} had no effect on gene expression). If $FC_a = FC_b$, this
% reflects a linear pathway, where \(A\) and \(B\) are both necessary for function in
% the pathway under study; this will correspond to an epistasis value of negative
% one half, $s(a, b) = -0.5$. If $\log{FC_a} < \log{FC_b}\), and the signs of
% $\log{FC_a}\) and $\log{FC_b}\) are the same, then the epistasis coefficient will
% have a value between $-1$ and $-0.5 $. On the other hand, if $\log{FC_a} >
% \log{FC_b}\) and both have the same sign, then the epistasis coefficient will lie
% between $-0.5$ and 0. These two cases correspond to branched pathways, where
% \gene{a} affects the transcriptional output via at least two different pathways,
% only some of which depend on \gene{b}. One notable characteristic of the
% epistasis coefficient is that for linear pathways, regardless of whether they
% are activating or inhibiting, the epistasis coefficient is guaranteed to be the
% same for all transcriptional elements downstream of the two genes under study.
%
% The epistasis coefficient, while a useful summary statistic, is not immune to
% pathologies. For example, if \gene{a} and \gene{b} have inverse effects on
% gene expression, such that $FC_a = FC_b^{-1}\), then the epistasis coefficient
% will become undefined. If Eq.\ref{eq:epistasis} holds, then we can still say
% that \gene{a} is epistatic over \gene{b}, but we should be wary of drawing a
% genetic pathway to explain this behavior, since we need more information. These
% examples help illustrate how the epistasis coefficient should be used. To
% perform an epistatic analysis using a transcriptional reporter, confirmation
% that Eq.\ref{eq:epistasis} holds is vital. Having confirmed that this identity
% applies, the epistasis coefficient can be computed to gain information into
% the nature of the interaction between the two genes in question (activating or
% inhibiting; linear or branched).

\subsection*{The requirements for epistatic analysis from a statistical
             mechanical perspective}
\subsubsection*{Null alleles.}
Geneticists have known for a long time that epistatic analyses must be carried
out using verified null alleles (alleles that do not generate product, or which
product is devoid of all biochemical activity). The reason for this requirement
becomes abundantly clear from Eq.~\ref{eq:epistasis2}. Since the genetic
activity of \gene{a} is modulated by \gene{b}, failure to use a null allele of
\gene{a} means that the loss of the activity provided by \gene{b} in the double
mutant will result in a measurable decrease of the activity of the gene product,
\(A\). Thus, the \Freg{} factor in the double mutant will not be the same as the
\Freg{} factor in the mutant of \gene{a}, and the equality requirement for
epistasis will be violated. Notably, a null allele of \gene{b} is not essential,
strictly speaking, to detect epistasis. However, failure to obtain a null allele
of \gene{b} will inhibit our ability to test whether the pathway via which these
genes interact is branched or not (see below).

\subsubsection*{Sensitization.}
Another difficulty with genetic analysis of a pathway can occur if the genes
under study are not major determinants of the phenotypic outcome. This is
equivalent to stating that \Freg{} depends on \(A\), \(B\) and at least one
other factor, \(C\); and that \(p_{bound}(A, B, C) \sim p_{bound}(0, 0, C)\).
The only case when Eq.~\ref{eq:pbound} suggests this can happen (in the absence
of measurement error) is when the effective polymerase activity, \(\rho\cdot
F_{reg}(0, 0, C)\) is very large. To study the functions of \gene{a} and
\gene{b} productively, it is necessary to re-scale \Freg{} to an appropriate
dynamic range such that \(p_{bound}\) is elastic to changes in the gene
activities of \gene{a} and \gene{b}. Since \Freg{} depends on the gene
activity of \gene{c}, the correct approach is to use an allele of \gene{c} that
decreases \Freg{} towards the desired direction. It is imperative that a
null allele of \gene{c} \textbf{not} be used, since \gene{c} may be epistatic to
\gene{a}, \gene{b} or both, in which case we would be unable to study
interactions between \gene{a} and \gene{b} in this genetic background. An
alternative to this strategy would be to decrease \(\rho\), but since RNA
polymerase is a crucial component of all cells, this is not advisable.

\subsection*{Genetic pathways in statistical mechanics}

\begin{figure}
  \includegraphics[width=\linewidth]{theoryims/Z_to_Genetics.pdf}
  \caption{Promoter architectures can be represented in three different ways.
  A promoter architecture can be represented by showing all the interactions
  between components simultaneously. An alternative representation is via
  genetic diagrams. The genetic diagram obviates the epistatic element. Finally,
  the statistical mechanical view is an enumeration of all the possible states
  in the promoter. In the above figure, only the active states are shown for
  brevity. A state is considered active if RNAPII is bound to the promoter.
  }\label{fig:Z_to_Genetics}
\end{figure}

Genetic diagrams are representations that satisfy observed epistasis
relationships between genes. Since we have found a mapping between epistasis
relationships and statistical mechanics, we can find the genetic pathway
equivalent to the states oriented picture of promoter architectures from
statistical mechanics (see Fig.~\ref{fig:Z_to_Genetics}). As a representative
example, envision a promoter architecture where there is an activator, \(A\),
which recruits RNA polymerase to the promoter and which can bind DNA with a
specific affinity; a helper or tethering protein, \(H\), that can bind the
activator as well as DNA, but cannot bind RNA polymerase; and RNA polymerase
present at some concentration. In this system, the helper molecule is not
necessary for activator function. The \Freg{} function for this system takes
the following analytical form:

\begin{equation}
    F_{reg}(A, H) = \frac{1 + A e^{-\varepsilon _{AP}}\frac{1 + H e^{-\varepsilon _{AH}}}
                                                       {1 + H}
                          }
                          {1 + A\frac{1 + H e^{-\varepsilon _{AH}}}
                                     {1 + H} }.
\label{eq:AH}
\end{equation}
Here, \(A\) and \(H\) represent the amount of activator and helper; \(\varepsilon _{AH}\)
is the energy of binding of activator to helper; and \(\varepsilon _{AP}\) is the
energy of binding of activator to polymerase. For simplicity, energies are
expressed in units of \(k_B T\).

As a first step, we confirm that this function satisfies the nested function
requirement for epistasis. We could re-write the above equation into a nested
function as follows:

\begin{equation}
    F_{reg}(A\cdot G_{reg}(H)) = \frac{1 + A e^{-\varepsilon _{AP}} \cdot G_{reg}(H)}
                                 {1 + A\cdot G_{reg}(H)}.
\label{eq:AHsimp}
\end{equation}

The fact that we can write this nested function means that the activator gene
is epistatic over the helper gene. Next, we will order the mutants according to
the magnitude of their \Freg{} factor. Since this is a system of activators,
the wild-type \(F_{reg}(A, B)\) must be greater than the value in any of the
mutants. Since \gene{a} is epistatic over \gene{b}, we know that \Freg{} for
the single mutant of \gene{a} and the double mutant have to be the same.
We also know that \Freg{} always increases with increasing effective gene
activity of \gene{a}, \(A\cdot G_{reg}(H)\). Finally, from inspection of
Eq.~\ref{eq:AH}, \(G_{reg}(0) = 1\). Putting all of this together, we can order
the factors:

\begin{equation}
F_{reg}[0] \leq F_{reg}[A \cdot G_{reg}(0)] \leq F_{reg}[A \cdot G(H)].
\end{equation}

This means that the phenotype of the double mutant \genotype{ab} and
\genotype{a} are of equal severity, and show the greatest perturbation relative
to the wild-type, while the single mutant of \gene{h} has a phenotype
intermediate to the phenotype of mutants of \gene{a} and the wild-type control.
Since the epistasis coefficient depends only on the \Freg{} associated with
each mutant, it follows that if we study how flexible this hierarchy can be,
we can study how this hierarchy can be modified by tuning the parameters in
Eq.~\ref{eq:AH}, and how this in turn can affect the value of the epistasis
coefficient.

We would like to know when
\(F_{reg}(A\cdot G_{reg}(0)) \sim F_{reg}[A \cdot G(H)]\).  This is equivalent to
stating that loss of \gene{h} does not measurably affect gene expression. This
will happen in the limit of saturating gene activity of \gene{a}, which results
from the combination of protein copy number, DNA binding affinity, and affinity
for RNA polymerase.
% This limit corresponds to an epistasis coefficient of 0,
% since the fold-change measured in null mutants of \gene{h} will be equal to 1.
% In this limit,
An epistatic analysis will conclude that \gene{a} and \gene{h}
do not interact along this phenotype and would identify a single pathway,
involving the activator gene, \gene{a}.

Another relevant limit is \(F_{reg}(0) \sim F_{reg}(A\cdot G_{reg}(0))\). For this
to be true, two conditions must hold. First, the gene activity of \gene{a} must
be low. Either the protein product is present at very low copy number or the
DNA binding affinity of the product is very low by itself. Second, the gene
activity of \gene{h} must not be low: It must have a reasonable combination of
protein copy number, DNA binding affinity and binding affinity to the product of
\gene{a} such that its regulatory function, \(G_{reg}(H)\) has a value much
greater than unity. In this regime, the single mutants of \gene{a} and \gene{h},
as well as the double mutant, would have the same fold-change relative to the
wild-type.
% In this case, the epistasis coefficient achieves a value of $-0.5$,
% which corresponds to a linear pathway.
This reflects the fact that the only
complex making a measurable contribution to polymerase binding is the complex
where the activator is bound to the helper. In other words, these two genes act
within a single, unbranched pathway that involves both \gene{a} and \gene{h}.

In between these two regimes, epistasis indicates the existence of two pathways,
one of which involves both \gene{a} and \gene{h}, and one of which involves only
\gene{a}. These pathways reflect the importance of an activator-polymerase
complex that does not include the helper protein; genetically, this secondary
complex constitutes a secondary, helper-independent pathway. The relative
importance of each pathway will depend strongly on the wild-type dosage levels.
This shows that statistical mechanical microstates can be mapped to genetic
pathways involving multiple genes. This mapping is not guaranteed to be
one-to-one (the mappings will usually be many-to-one), nor is it guaranteed to
describe all the states available to the pathway.

Although at no point is the equality in Eq.~\ref{eq:epistasis} violated, the
interpretation of the epistasis relationship is highly dependent on gene dosage.
Epistasis relationships are not immutable with dosage, and a given relationship
may change from independent to additive to suppressive along a given dosage
curve. This highlights the importance of varying the reference gene activity
levels using partial loss-of-function alleles to establish that
Eq.~\ref{eq:epistasis} holds along an entire curve.

\subsection*{Genetic Morphs}
Genetics can generate alleles with vastly different properties. Our formalism
allows us to immediately derive the most commonly encountered allele classes
(see Supplementary Information Section XXX: Common allelic classes and their
statistical mechanical definitions). In the following text, we assume the gene
in question promotes transcription; for an inhibitor, the definitions are the
same except that the effects on active and inactive states are flipped. An
active state is a state that has RNA polymerase bound.

\begin{itemize}
  \item \emph{Hypomorph}. \textbf{Genetic definition}: an allele with reduced
  gene activity, either by reduced product copy number or decreased biochemical
  activity, causing a loss-of-function mutant phenotype. \textbf{Statistical
  mechanical definition}: an allele which decreases the relative proportion of
  active states compared with the wild type homozygote.
  \item \emph{Hypermorph}. \textbf{Genetic definition}: an allele with increased
  gene activity, either by increased copy number or improved biochemical
  activity, causing an increased-function mutant phenotype; the hypermorphic
  phenotype can be phenocopied by overexpression of the wild type allele.
  \textbf{Statistical mechanical definition}: an allele that increases the
  relative proportion of active states compared with the wild type homozygote.
  \item \emph{Neomorph}. \textbf{Genetic definition}: an allele which has a
  novel functionality causing a mutant phenotype which cannot be phenocopied
  simply by overexpressing wild type product. The neomorphic allele generates
  product at a similar rate as the wild type allele. \textbf{Statistical
  mechanical definition}: an allele encoding a modified \Freg{} function,
  generating novel states (active or inactive) not accessible to the wild type
  product at any concentration.
\end{itemize}

Grouping alleles into the correct change-of-function classes is an important
part of genetic analysis because hypermorphic alleles from one gene can be used
in combination with null alleles at a second locus to order genes in a pathway.
However, neomorphic and hypermorphic alleles can be extremely difficult to
discern and can confound genetic analysis. From a statistical mechanical
perspective, this confounding arises because the neomorphic allele
changes the functional form of the interactions, which qualitatively alters
the system by adding neofunctionalized states, whereas the hypermorphic allele
exclusively changes the gene dosage relative to the wild-type levels.

\section*{Discussion}
We have shown that genetics is a method to study partition
functions that are not known analytically. The method relies on systematically
setting component values to zero, individually and pairwise, and identifying
circumstances when the partition function exhibits epistasis. Although equality
between two sets of perturbations can occur as a coincidence, the stringent
requirement of equality makes this an unlikely occurrence. Therefore, if
there is epistasis between two components, it is reasonable to infer that these
perturbed variables function as a single variable equal to a product of two
functions that each take as input one of the two component. This methodology is
known in genetics as epistatic analysis, and it provides a way to study, in an
analog fashion, a molecular system where the components and interactions may be
entirely unknown.

The results from epistatic analyses can be represented as genetic diagrams,
where the genes represent a component in the system and arrows indicate
interactions. Because of the analog nature of the analysis (components are
either `ON' or `OFF'), genetic diagrams do not have to yield information about
the dynamical capacities of the system, nor do they represent direct biochemical
interactions. Rather, genetic pathways are a graphical method of representing
epistatic interactions between components in a circuit, allowing geneticists to
rapidly identify the epistatic and hypostatic components in any given
comparison, and the weighting given to different arrows depends strongly on the
dosage of all the components connected by them. When these pathways are drawn to
represent elements of a promoter, pathways can be mapped onto subsets of active
states involving the elements connected by arrows.

Gene expression levels are increasingly being used as a phenotype for genetic
analysis. A particularly attractive feature of these phenotypes is our ability
to measure expression levels in a highly multiplexed fashion in a single
experiment, but the resulting datasets have proven hard to analyze due to their
high-dimensional nature. Dimensionality reduction methods such as principal
component analysis or non-negative matrix factorization have become popular ways
to analyze transcriptomic data. However, a drawback of these methods is that
they are derived from a statistical framework without connection to the
underlying biology, and the transformations that they apply to the data make it
difficult to interpret signals in terms of biological components. Another
powerful approach is to fit general linear models with interactions terms to
transcriptomic dataset. In this framework, the expression level of an individual
transcript is modeled with two first order coefficients that quantify the
independent effect of each component on the expression level of that transcript,
and an interaction term that is non-zero when the double mutant shows
non-additive expression levels. Here, we argue that this approach does not
extract all the available information encoded in transcriptomes because it does
not, on its own, identify cases where classical epistasis is occurring.

We have shown that, for systems at equilibrium, epistasis arises naturally.
Moreover, we have shown that epistasis percolates through a network at
equilibrium regardless of its depth, and we speculate whether this result may
explain the unreasonable effectiveness of genetics at enormous differences in
scale, since macroscopic (organismal or population) phenotypes to microscopic
phenotypes all show epistasis. However, a major caveat to this work is the
fact that biological systems are not at thermal equilibrium. The commonality of
classical epistasis in biological systems suggests that many of our results will
have a non-equilibrium analog with some changes. Certainly, networks that are
not at equilibrium will not reveal epistasis at arbitrary depth. For
non-equilibrium networks, it will be particularly interesting to study how
epistasis is affected by relaxing the acyclic requirement we imposed. Feedback
might be expected to hide epistasis; however, biological experiments suggest
that certain topologies, even with feedback, show epistasis robustly. The study
of the physical basis of epistasis in biological systems seems poised to reveal
fascinating new insights into the principles of how these networks are built.

  \printbibliography[heading=subbibliography]
\end{refsection}

\chapter{Reconstructing a metazoan genetic pathway with transcriptome-wide
         epistasis measurements}
\begin{refsection}
  \input{chapters/hypoxia.tex}
  \printbibliography[heading=subbibliography]
\end{refsection}

\chapter{The \emph{Caenorhabditis~elegans} Female-Like State: Decoupling the
         Transcriptomic Effects of Aging and Sperm Status}
\begin{refsection}
  \input{chapters/female-transcriptome.tex}
  \printbibliography[heading=subbibliography]
\end{refsection}

\chapter{Using transcriptomes as mutant phenotypes reveals functional regions of
         a Mediator subunit in \cel{}}
\begin{refsection}
  \newcommand{\gf}{gain-of-function allele}
\newcommand{\lf}{loss-of-function allele}
\newcommand{\strong}{strong loss-of-function allele}
\newcommand{\weak}{weak loss-of-function allele}

% gene names
\newcommand{\ras}{\gene{let-60} (\emph{ras})}
\newcommand{\rasp}{\protein{let-60}}
\newcommand{\dpy}[1]{\gene{dpy-22#1}}
\newcommand{\letgfn}{3,021}
\newcommand{\letlfn}{857}
\newcommand{\letgf}{\gene{let-60(gf)}}
\newcommand{\letlf}{\gene{let-60(lf)}}
\newcommand{\strongn}{2,036}
\newcommand{\weakn}{266}
\newcommand{\transn}{2,128}
\newcommand{\bx}{\dpy{(bx93)}}
\newcommand{\sy}{\dpy{(sy622)}}

% more space between rows
\newcommand{\ra}[1]{\renewcommand{\arraystretch}{#1}}

% document begins here
\section*{Abstract}
  \textbf{Although transcriptomes have recently been used as phenotypes with
  which to perform epistasis analyses, they are not yet used to study intragenic
  function/structure relationships. We developed a theoretical framework to
  study allelic series using transcriptomic phenotypes. As a proof-of-concept,
  we apply our methods to an allelic series of \dpy{}, a highly pleiotropic
  \emph{Caenorhabditis~elegans} gene orthologous to the human gene \gene{MED12},
  which encodes a subunit of the Mediator complex. Our methods identify
  functional units within \dpy{} that modulate Mediator activity upon various
  genetic programs, including the Wnt and Ras modules.}

\section*{Introduction}
Mutations of a gene can yield a series of alleles with different phenotypes that
reveal multiple functions encoded by that gene, regardless of the alleles'
molecular nature. In \emph{Caenorhabditis~elegans}, allelic series have
characterized genes such as \gene{let-23/EGFR}, \gene{lin-3/EGF} and
\gene{lin-12/NOTCH}~\citep{Aroian1991, Ferguson1985a, Greenwald1983}. Allelic
series provide a way to probe genes where biochemical approaches would be
difficult, slow or uninformative with regards to the biological phenomenon of
interest. Their power derives from the ability to draw broad conclusions about
the gene of interest in terms of gene dosage and functional units, to the extent
that these two factors are separable, without regard to the molecular identity
of the mutations that created these alleles. Here, gene dosage is defined as the
combined effects of transcriptional and translational expression, gene product
localization, and biochemical kinetics of the final gene product \emph{in situ}.
To study allelic series, we must first enumerate the phenotypes each allele
affects, and subsequently order the alleles into severity and dominance
hierarchies per phenotype. The resulting hierarchies enable us to better
understand how a given gene, which may be highly pleiotropic, can give rise to
highly specific mutant phenotypes when mutated in just the right way.

Biology has moved from expression measurements of single genes towards
genome-wide measurements. Expression profiling via RNA-seq~\citep{Mortazavi2008}
enables simultaneous measurement of transcript levels for all genes in a genome,
yielding a transcriptome. These measurements can be made on whole organisms,
isolated tissues, or single cells~\citep{Tang2009,Schwarz2012}. Transcriptomes
have been successfully used to identify new cell or organismal
states~\citep{Angeles-Albores2017,Villani2017}. Transcriptomic states can be
used to perform epistatic analyses~\citep{Dixit2016,AngelesAlboresHIF},
but have not been used to characterize allelic series.

We have devised methods for characterizing allelic series using RNA-seq. To test
these methods, we selected three alleles~\citep{Zhang2000,Moghal2003} of a
\cel{} Mediator complex subunit gene, \dpy{}. Mediator is a macromolecular
complex with $\sim25$ subunits~\citep{Jeronimo2017} that globally regulates RNA
polymerase II (Pol II)~\citep{Allen2015,Takagi2006}. The Mediator complex has at
least four biochemically distinct modules: the Head, Middle and Tail modules and
a CDK-8-associated Kinase Module (CKM). The CKM associates reversibly with other
modules, and appears to inhibit transcription~\citep{Knuesel2009,Elmlund2006}.
In \cel{} development, the CKM promotes the formation of the male
tail~\citep{Zhang2000} (through interactions with the Wnt pathway), as well as
formation of the hermaphrodite vulva~\citep{Moghal2003} (through inhibition of
the Ras pathway). Null alleles of \dpy{} are likely to be lethal, based on
embryonic lethal phenotypes observed after RNAi~\citep{Wang2004,Lehner2006} and
the severe phenotypes of a strong \dpy{} hypomorphic allele, \gene{dpy-22(e652)}
(homozygous hermaphrodites are very sick)~\citep{Riddle1997}. Homozygotes of
allele \gene{dpy-22(bx93)}, which encodes a premature stop codon
Q2549Amber~\citep{Zhang2000}, appear grossly wild-type, though this allele does
not have complete wild-type functionality, since it fails to fully complement
the Muv phenotype of another allele, \emph{sy622}, in a sensitized \emph{let-23}
background. In contrast, animals homozygous for a more severe allele,
\gene{dpy-22(sy622)} encoding another premature stop codon,
Q1698Amber~\citep{Moghal2003}, are dumpy (Dpy), have egg-laying defects (Egl),
and have multiple vulvae (Muv) (Fig.~\ref{fig:genemodel}). In
humans, \protein{MED12} is known to have a proline-, glutamine- and leucine-rich
domain that interacts with the WNT pathway~\citep{Kim2006}. However, many
disease-causing variants fall outside of this domain~\citep{Yamamoto2015}.
In spite of its causative role in a number of neurodevelopmental
disorders~\citep{Graham2013}, the structural and functional features of this
gene are poorly understood, partially because genetic approaches towards
studying pleiotropic genes have proved difficult in the past, highlighting the
need for new methods.

\begin{figure}
  \centering{}
  \includegraphics[width=0.5\textwidth]{alleleims/Gene_Model.pdf}
  \caption{
           Protein sequence schematic for \protein{dpy-22}. The positions of the
           nonsense mutations used are shown.
           }
\label{fig:genemodel}
\end{figure}


\section*{Methods}\label{sec:methods}

\subsection*{Strains used}
Strains used were N2 wild-type (Bristol)~\citep{Brenner1974},
PS4087 \sy{}~\citep{Moghal2003},
PS4187 \bx{}~\citep{Zhang2000},
PS4176 \gene{dpy-6(e14) dpy-22(bx93)/+ dpy-22(sy622)}~\citep{Moghal2003},
MT4866 \gene{let-60(n2021)}~\citep{Beitel1990a},
MT2124 \gene{let-60(n1046gf)}~\citep{Beitel1990a} and
EW15 \gene{bar-1(ga80)}~\citep{Eisenmann1998}.
Lines were grown on standard nematode growth media (NGM) Petri plates seeded
with OP50 \ecol{} at 20\degree{}C~\citep{Brenner1974}.

\subsection*{Strain synchronization, harvesting and RNA sequencing}
With the exception of strain MT4866, strains were synchronized by bleaching
P$_0$'s into virgin S. basal (no cholesterol or ethanol added) for 16--18 hours.
Arrested L1 larvae were placed in NGM plates seeded with OP50 at 20\degree{}C
and grown to the young adult stage (assessed by vulval morphology and lack of
embryos). We discovered that MT4866 dies upon L1 starvation for this period of
time. As a result, we synchronized this strain by double bleaching. Animals were
picked if they were young adults, regardless of whether any vulval or
morphological phenotypes were present. RNA extraction and sequencing was
performed as previously described by~\citet{AngelesAlboresHIF,
Angeles-Albores2017}. Briefly, young adults were placed in 10~$\mu$L of TE
buffer,  and digested using  Recombinant Proteinase K PCR Grade (Roche Lot 656
No. 03115 838001) incubated with 1\% SDS 657 and 1.25~$\mu$L RNA Secure (Ambion
AM7005). Total RNA was extracted using the Zymo Research Directzol RNA MicroPrep
Kit (Zymo Research, SKU R2061).\@ mRNA was subsequently purified using a NEBNext
Poly(A) mRNA Magnetic Isolation Module (New England Biolabs, NEB, \#E7490).
Sequencing libraries were generated using the NEBNext Ultra RNA Library Prep Kit
for Illumina (NEB \#E7530). These libraries were sequenced using an Illumina
HiSeq2500 machine in single-read mode with a read length of 50 nucleotides.

\subsection*{Read pseudo-alignment and differential expression}
Reads were pseudo-aligned to the \cel{} genome (WBcel235) using
Kallisto~\citep{Bray2016}, using 200 bootstraps and with the sequence bias
(\texttt{--seqBias}) flag. The fragment size for all libraries was set to 200
and the standard deviation to 40. Quality control was performed on a subset of
the reads using FastQC, RNAseQC, BowTie and MultiQC~\citep{Andrews2010,
Deluca2012, Langmead2009, Ewels2016}.

Differential expression analysis was performed using
Sleuth~\citep{Pimentel2016a}. We used a general linear model to identify genes
that were differentially expressed between wild-type and mutant libraries. To
increase our statistical power, we pooled young adult wild-type replicates from
other published~\citep{AngelesAlboresHIF, Angeles-Albores2017} and unpublished
analyses adjusting for batch effects. Briefly, batch effects were controlled by
including the identity of the person who collected the worms and the method
by which the libraries were generated as covariates.å

\subsection*{False hit analysis}
To accurately count phenotypes, we developed a false hit algorithm
(Algorithm~\ref{alg:false}). We implemented this algorithm for comparisons of
three genotypes using Python. Such an experiment can result in $128$ possible
combinations of phenotypic classes (ignoring size). This large number
of models necessitates an algorithmic approach that can restrict the
number of models. Our algorithm uses a noise function that assumes
false hit events are non-overlapping (i.e.\ the same gene cannot be the result
of two false positive events in two or more genotypes) to determine the average
noise flux between phenotypic classes. These assumptions break down if
false-positive or negative rates are large (>25\%).

To benchmark our algorithm, we generated one thousand Venn diagrams at random.
For each Venn diagram, we calculated the average false positive and false
negative flux matrices. Then, we added noise to each phenotypic class in the
Venn diagram, assuming that fluxes were normally distributed with mean and
standard deviation equal to the flux coefficient calculated. We input the noised
Venn diagram into our false hit analysis and collected classification
statistics. For a given signal-to-noise cutoff, $\lambda$, classification
accuracy varied significantly with changes in the total error rate. In the
absence of false negative hits, false hit analysis can accurately identify
non-empty genotype-associated phenotypic classes, but identifying
genotype-specific classes becomes difficult if the experimental false positive
rate is high. On the other hand, even moderate false negative rates ($>10\%$)
rapidly degrade signal from genotype-associated classes. For classes that are
associated with three genotypes, an experimental false negative rate of 30\% is
enough on average to prevents this class from being observed.

We selected $\lambda=3$ because classification using this threshold was high
across a range of false positive and false negative combinations. A challenge to
applying this algorithm to our data is the fact that the false negative rate for
our experiment is unknown. Although there has been significant progress in
controlling and estimating false positive rates, we know of no such attempts for
false negative rates. It is unlikely that the false negative rate for our study
is lower than the false positive rate, because all genotypes except the controls
are likely underpowered. We used false negative rates between 10--20\% for false
hit analysis. All analyses returned the same final model.

We asked whether re-classification of some classes into others could improve
model fit. We manually re-classified the (\sy{},\bx{})-associated and the (\bx,
\emph{trans-heterozygote})-associated classes into the \emph{bx93}-associated
class (which is associated with all genotypes), and compared $\chi^2$ statistics
between a re-classified reduced model ($\chi^2=72$) and a reduced model
($\chi^2=130$). Based on the lower $\chi^2$ of the re-classified reduced model,
we concluded that it is the most likely model given our data.

\begin{algorithm}[H]
  \DontPrintSemicolon{}
  \KwData{$\mathbf{M}_{obs} =  \{N_l\}$, an observed set of classes, where each
  class is labelled by $l\in L$ and is of size $N_l$. $f_p, f_n$, the false
  positive and negative rates respectively. $\alpha$, the signal-to-noise
  threshold for acceptance of a class.}
  \KwResult{$\mathbf{M}_{reduced}$, a reduced model that fits the data.}
  \BlankLine{}
  \Begin{
    \emph{Define a minimal model}, $\mathbf{K}$\;

    \emph{Refine the model until convergence or iterations max out}\; \\
    $i \leftarrow 0$\; \\
    $\mathbf{K_{prev}} \leftarrow \emptyset$\; \\
    \While{$(i < i_{\max})~|~(\mathbf{K_{prev}} \neq \mathbf{K}$)}{
      $\mathbf{K_{prev}} \leftarrow \mathbf{K}$\;

      \emph{Define a noise function to estimate error flows in
            $\mathbf{K}$}\;
      $\mathbf{F} \leftarrow \textrm{noise}(\mathbf{K}, f_p, f_n)$\;

      \For{$l \in L$}{
          \emph{Calculate signal to noise for each labelled class}\;
          \emph{False negatives can result in $\lambda < 0$}\;
          $\lambda_l \leftarrow \mathbf{M}_{obs, l}/F_{l}$\;
          % \emph{Use classes with high $\lambda_l$ to refine the model}\;
              \If{$(\lambda > \alpha)~|~(\lambda < 0)$}{
                $\mathbf{K}_l \leftarrow \mathbf{M}_{obs, l}$\;
              } %if
          } % end for
        $i++$
    } % end while
  } % end begin

  $\mathbf{M}_{reduced} = \mathbf{K}$\;

  \Return{$\mathbf{M}_{reduced}$}\;
  \BlankLine{}\;
  \caption{False Hit Algorithm. Briefly, the algorithm initializes a reduced
           model with the phenotypic class or classes labelled by the largest
           number of genotypes. This reduced model is used to estimate noise
           fluxes, which in turn can be used to estimate a signal-to-noise metric
           between observed and modelled classes. Classes that exhibit a high
           signal-to-noise are incorporated into the reduced model.}
  \label{alg:false}
\end{algorithm}


\subsection*{Dominance analysis}
\label{subsec:dominance}
We modeled allelic dominance as a weighted average of allelic activity:
\begin{equation}
  \beta_{a/b,i,\text{Pred}}(d_a) = d_a\cdot \beta_{a/a,i} +
                                   (1-d_a)\cdot \beta_{b/b,i},
\end{equation}
where $\beta_{k/k, i}$ refers to the $\beta$ value of the $i$th isoform in a
genotype $k/k$, and $d_a$ is the dominance coefficient for allele $a$.

To find the parameters $d_a$ that maximized the probability of observing the
data, we found the parameter, $d_a$, that maximized the equation:
\begin{equation}
    P(d_a|D,H,I) \propto \prod_{i \in S}
                   \exp{-\frac{{(\beta_{a/b,i,\text{Obs}} -
                                \beta_{a/b,i,\text{Pred}}(d_a))}^2}{
                                2\sigma_i^2}}
\end{equation}
where $\beta_{a/b,i,\text{Obs}}$ was the coefficient associated with the $i$th
isoform in the \emph{trans}-het $a/b$ and $\sigma_i$ was the standard error of
the $i$th isoform in the \emph{trans}-heterozygote samples as output by
Kallisto. $S$ is the set of isoforms that participate in the regression (see
main text). This equation describes a linear regression which was solved
numerically.

\subsection*{Code}
Code was written in Jupyter notebooks~\citep{Perez2007} using the Python
programming language. The Numpy, pandas and scipy libraries were used for
computation~\citep{VanDerWalt2011,McKinney2011,Oliphant2007} and the matplotlib
and seaborn libraries were used for data visualization~\citep{Hunter2007,
Waskom}. Enrichment analyses were performed using the WormBase Enrichment
Suite~\citep{Angeles-Albores2016,Angeles-Albores2018}. For all enrichment
analyses, a $q$-value of less than $10^{-3}$ was considered statistically
significant. For gene ontology enrichment analysis, terms were considered
statistically significant only if they also showed an enrichment fold-change
greater than 2.

\subsection*{Data Availability}
Raw and processed reads were deposited in the Gene Expression Omnibus. Scripts
for the entire analysis can be found with version control in our Github
repository, \url{https://github.com/WormLabCaltech/med-cafe}. A user-friendly,
commented website containing the complete analyses can be found at\\ % formatting
\url{https://wormlabcaltech.github.io/med-cafe/}. Raw reads and quantified
abundances for each sample were deposited at the NCBI Gene Expression Omnibus
(GEO)~\citep{Edgar2002} under the accession code GSE107523
(\url{https://www.ncbi.nlm.nih.gov/geo/query/acc.cgi?acc=GSE107523}).

\section*{Results}
\subsection*{RNA-sequencing of three \gene{dpy-22} alleles and two known
             interactor genes}
We carried out RNA-seq on biological triplicates of mRNA extracted from \sy{}
homozygotes, \bx{} homozygotes, and wild type controls, along with
quadruplicates from \emph{trans}-heterozygotes of both alleles with the genotype\\
\gene{dpy-6(e14) dpy-22(bx93)/+ dpy-22(sy622)}. We also sequenced mRNA extracted
from \gene{bar-1(ga80)} (the $\beta$-catenin ortholog in \cel{}),
\gene{let-60(n2021)} and \gene{let-60(n1046gf)} (the Ras ortholog in \cel{})
mutants in triplicate because these genes have been previously described to
interact with \dpy{} to form the vulva~\citep{Moghal2003} and the male
tail~\citep{Zhang2000}. Sequencing was performed at a depth of 20 million reads
per sample. Reads were pseudoaligned using Kallisto~\citep{Bray2016}. We
performed a differential expression using a general linear model specified using
Sleuth~\citep{Pimentel2016a} (see~\hyperref[sec:methods]{Methods}). Differential
expression with respect to the wild type control for each transcript $i$ in a
genotype $g$ is measured via a coefficient $\beta_{g, i}$, which can be loosely
interpreted as the natural logarithm of the fold-change. Transcripts were
considered to have differential expression between wild-type and a mutant if
their false discovery rate, $q$, was less than or equal to 10\%. We used this
method to identify the differentially expressed genes associated with each
mutant (Table~\ref{tab:numbers};
\href{https://wormlabcaltech.github.io/med-cafe/notebook/basic.html}{Basic
Statistics Notebook}) Supplementary File 1 contains all the beta values
associated with this project. We have also generated a website containing
complete details of all the analyses available at the following URL:\@
\url{https://wormlabcaltech.github.io/med-cafe/analysis}.

\begin{table*}
 \centering
 \begin{tabular}{lc}
   \toprule
   Genotype & Differentially Expressed Genes\\
   \midrule
   \bx{} & \weakn{}\\
   \gene{dpy-6(e14)} \dpy{(bx93)} / \emph{+} \dpy{(sy622)} & \transn{}\\
   \sy{} & \strongn{}\\
   \gene{bar-1(ga80)} & 4613\\
   \gene{let-60(n2021)} & 509\\
   \gene{let-60(n1046gf)} & 2526\\
   \bottomrule
 \end{tabular}
 \caption{
          The number of differentially expressed genes relative to the wild-type
          control for each genotype with a significance threshold of 0.1.
          }
\label{tab:numbers}
\end{table*}


\begin{figure}
  \centering{}
  \includegraphics[width=0.45\textwidth]{alleleims/pca.pdf}
  \caption{
           Principal component analysis of the analyzed genotypes. The
           analysis was performed using only those transcripts that were
           differentially expressed in at least one genotype. The plot shows
           that the \emph{trans}-heterozygotes phenocopy the \bx{} homozygotes
           along the first two principal dimensions.
           }
\label{fig:allelic_pca}
\end{figure}

\subsection*{Principal component analysis visualizes the allelic dominance of
             the \bx{} allele over \sy{}}
As a first step in our analysis, we performed dimensionality reduction on the
transcriptomes we sequenced using Principal Component Analysis (PCA). Briefly,
PCA identifies the vectors along which there is most variation in the data.
These vectors can be used to project the data into lower dimensions to assess
whether samples cluster, though interpreting the biological reasons for this
clustering can be challenging. To perform PCA, we selected only those
transcripts that were differentially expressed in at least one genotype, and
used the $\beta$ coefficients associated with these genes to perform PCA.\@
Projecting the data into two dimensions maintains 65\% of the variation. The
first dimension separates the gain and loss of function \gene{let-60} mutants.
The second dimension separates the \dpy{} mutants (Fig.~\ref{fig:allelic_pca}). On the
PCA plot, the \emph{trans}-heterozygote mutants appear to phenocopy the \bx{}
mutants, recapitulating previous experiments that showed the \bx{} allele to be
dominant over the \sy{} allele.

\subsection*{Three \dpy{} genotypes have shared transcriptomic
             phenotypes}
We would like to understand the degree and nature of the dominance
between these \dpy{} alleles. To construct a severity and dominance hierarchy,
we must establish how many transcriptomic phenotypes are represented among the
three \dpy{} genotypes, and of those phenotypes, how many of them are shared
transcriptomic phenotypes (STPs). Shared transcriptomic phenotypes are defined
as the set of genes that are commonly differentially expressed in two mutant
genotypes relative to a wild-type control, regardless of the direction of
change, as defined previously~\citep{AngelesAlboresHIF}. We use the term in the
plural version, because the shared genes may represent multiple independent
modules that formally constitute different phenotypic classes.

We identified significant pairwise STPs between all \dpy{} mutants. The
transcripts that were differentially expressed in \bx{} homozygotes were almost
all differentially expressed in \sy{} homozygotes (189/\weakn{}) and in
\emph{trans}-heterozygotes (192/\weakn{}). On the other hand, although \sy{}
homozygotes and \emph{trans}-heterozygotes exhibited a similar number of
differentially expressed genes, less than half of these were shared between the
two genotypes.

\subsection*{False hit analysis identifies four non-overlapping phenotypic
             classes}

\begin{figure*}
 \centering{}
 \includegraphics[width=\textwidth]{alleleims/flowchart.pdf}
 \caption{
         Flowchart for an analysis of arbitrary allelic series. A set of alleles
         is selected, and the corresponding genotypes are sequenced. Independent
         phenotypic classes are identified, and classes that are the result of
         noise are discarded via a false hit analysis. For each phenotypic
         class, the alleles are ordered in a dominance/complementation
         hierarchy, which can then be used to infer functional regions (FR)
         within the genes in question.
         }
\label{fig:flowchart}
\end{figure*}

Severity and dominance hierarchies must be calculated with respect to each
independent phenotype associated with the alleles under study. A challenge with
expression profiles is to identify these independent phenotypes. We reasoned
that comparing the expression profiles of the two \dpy{} homozygotes and the
\emph{trans}-heterozygote would naturally partition the expression profiles into
groups that would constitute phenotypic classes. However, a three-way comparison
can give rise to 7 ($2^3-1$) possible groupings: transcripts perturbed in only a
single genotype (3), transcripts perturbed in two genotypes (3) and transcripts
perturbed in all three genotypes (1). A shortcoming of RNA-seq is that it is
prone to false positive and false negative artifacts, and these artifacts could
be numerous enough to cause the appearance of certain groups that would not
be there otherwise. In other words, we might find a subset of genes that are
differentially expressed in a single genotype, but if this subset is small
enough, we ought to be concerned that this subset is caused by false positive
hits within this genotype or false negative hits in the other genotypes. This
thought experiment highlights the need to assess which groups have sufficient
statistical support to consider as phenotypic classes.

We developed a method to assess whether groups in a Venn diagram are likely to
be the result of statistical artifacts. Briefly, the algorithm works by first
assuming all of the data is the result of false positive and false negative hits
except for the group of transcripts that is differentially expressed in most
genotypes. Then, using estimates for the false positive and negative response,
we calculate the expected sizes of all the groups after adding noise under this
model. If an observed group is much larger than expected by noise, we refine the
data model to accept the group. This process is iterated until the data model
converges. We called this method a false hit analysis.

We used false hit analysis to identify four non-overlapping phenotypic classes
(Fig.~\ref{fig:flowchart}). We use the term genotype-specific to refer to groups
of transcripts that were perturbed in one mutant genotype. We use the term
genotype-associated to refer to those groups of transcripts whose expression was
significantly altered in two or more mutants genotypes with respect to the wild
type control. The \textbf{\sy{}-associated} phenotypic class consisted of 665
genes differentially expressed in \sy{} homozygotes and in
\emph{trans}-heterozygotes, but which had wild-type expression in \bx{}
homozygotes. The \textbf{\bx{}-associated} phenotypic class contains 229 genes
differentially expressed in all genotypes. The \bx{}-associated class included
re-classified transcripts that had been found to be differentially expressed in
the \bx{} homozygote and one other genotype, because these were very likely to
be the result of false negative hits in the missing genotype, and re-classifying
these transcripts improved our model substantially. We also identified a
\textbf{\sy{}-specific} phenotypic class (1,213 genes) and a
\textbf{\emph{trans}-heterozygote-specific} phenotypic class (1,302 genes; see
the
\href{https://wormlabcaltech.github.io/med-cafe/notebook/phenotypic_classes.html}{
Phenotypic Classes Notebook}).

\subsection*{Severity hierarchy of a \gene{dpy-22} allelic series}
Having separated the expression profiles into phenotypic classes, we can ask
what the severity hierarchy is between the \bx{} allele and the \sy{} allele.
Broadly speaking, there are two ways to assess severity. First, we can ask which
allele causes more mutant phenotypes or phenotypic groups as a homozygote
(\textbf{allelic pleiotropy}). Alternatively, we can identify the allele which
causes the greatest change in expression in a homozygote at each shared
phenotype among the homozygotes of both alleles, which we refer to as
\textbf{allelic magnitude}. An important caveat is that magnitude only makes
sense if the homozygotes of each allele are well correlated (i.e., they have a
linear relationship with small spread). If the phenotypes have zero or negative
correlation between two homozygotes, then the two alleles under inspection are
not of the same kind, i.e., they cannot both be loss-of-function alleles or
gain-of-function alleles for this phenotype, though the converse is not
necessarily true.

The \sy{} homozygote shows more differentially expressed genes that participate
in a greater number of phenotypic classes relative to the \bx{} homozygote.
Thus, the \sy{} allele is a more pleiotropic mutation than the \bx{} allele.
Since the homozygotes of each allele only share a single phenotypic class in
common, we need only assess magnitude along this single phenotype. To calculate
a magnitude coefficient, for genes in the \bx{}-associated phenotypic class, we
plotted the $\beta$ coefficients from the \sy{} homozygote against the $\beta$
coefficients from the \bx{} homozygote (see Fig.~\ref{fig:stp}) and performed a
linear regression to find the slope of this line. Using this method, we found
that the \bx{} homozygote has a magnitude that is 62\% $\pm2\%$ of the \sy{}
homozygote. Taken together, these results suggest that the \sy{} allele
represents a more severe alteration-of-function mutation than the mutation
within the \bx{} allele.

\begin{figure}
  \includegraphics[width=0.5\textwidth]{alleleims/dpy22-stps.pdf}
  \caption{
           Shared Transcriptomic Phenotypes amongst the \dpy{} genotypes are
           regulated in the same direction. For each pairwise comparison, we
           found those transcripts that were commonly differentially expressed
           in both genotypes relative to the wild-type control and plotted the
           $\beta$ coefficients for each. We performed a linear regression on
           each plot to find the line of best fit (broken blue line). Only the
           comparison between \sy{} and \bx{} homozygotes was used to establish
           that the magnitude of the \sy{} allele is greater than the magnitude
           of the \bx{} allele. The other comparisons are shown for
           completeness.
          }
\label{fig:stp}
\end{figure}

\subsection*{Dominance hierarchy of a \gene{dpy-22} allelic series}
We measured allelic dominance for each class using a dominance coefficient
(see~\hyperref[sec:methods]{Methods}). The dominance coefficient is a measure of
the contribution of each allele to the total expression level in
\emph{trans}-heterozygotes. By definition, the \sy{} allele is completely
recessive to \bx{} for the \sy{}-specific phenotypic class. To determine the
dominance coefficient for the remaining phenotypic classes, we first selected
the transcripts within those classes, and asked what linear combination of the
homozygotic $\beta$ coefficients best approximated the $\beta$ coefficients of
the \emph{trans}-heterozygote, subject to the constraint that the sum of the
weights for the two homozygotes should be equal to unity. We solved this problem
by finding the maximum likelihood estimate for these weights. Using this method,
we found that the \sy{} and \bx{} alleles are semidominant ($d_{bx93} = 0.48$)
to each other for the \sy{}-associated phenotypic class. The \bx{} allele is
largely  dominant over the \sy{} allele ($d_{bx93}=0.82$; see
Table~\ref{tab:dom}) for the \bx{}-associated phenotypic class.

\begin{table}
  \centering
  \begin{tabular}{lc}
    \toprule
    Phenotypic Class & Dominance\\
    \midrule
    \sy{}-specific & $1.00\pm0.00$\\
    \sy{}-associated & $0.48\pm0.01$\\
    \bx{}-associated & $0.82\pm0.01$\\
    \bottomrule
    % \midrule{}
  \end{tabular}
  \caption{
           Dominance analysis for the \dpy{/MDT12} allelic series. Dominance
           values closer to 1 indicate \bx{} is dominant over \sy{}, whereas 0
           indicates \sy{} is dominant over \bx{}.
          }
\label{tab:dom}
\end{table}

\subsection*{Phenotypic classes reflect morphological phenotypes}
We performed enrichment analysis of anatomical, phenotypic and gene ontology
terms using the WormBase Enrichment Suite~\citep{Angeles-Albores2016,
Angeles-Albores2018}. The \bx{}-associated phenotypic class was enriched in
genes involved in `immune system processes' (\qval{5}), and was enriched in
genes expressed in the `intestine' (\qval{4}). The \sy{}-associated
class was enriched in genes expressed in the `cephalic sheath
cell' (\qval{4}). Using ontology enrichment analysis from the WormBase
Enrichment Suite, we found that the \sy{}-associated class is enriched in
histones and histone-like proteins (`DNA packaging complex' \qval{3}) as well as
genes involed in `immune system processes' (\qval{5}). The \sy{}-specific class
was enriched in genes that have expression in the `intestine' (\qval{7}),
`muscular system' (\qval{3}) and `epithelial system' (\qval{2}). The genes in
this class are known to cause bacterial lawn avoidance when knocked down or
knocked out (\qval{2}). Finally, GO enrichment showed that the \sy{}-specific
class is specifically enriched in `structural constituents of cuticle'
(\qval{12}), and in genes involved in respiration (\qval{6}). This last result
recapitulates the fact that \sy{} homozygotes show a severe Dumpy phenotype. The
\emph{trans}-heterozygote specific class was enriched in genes expressed in
`male' animals (\qval{63}) and genes expressed in the `reproductive system'
(\qval{21}). GO enrichment of genes in the \emph{trans}-heterozygote specific
class showed enrichment of the genes involved in the `regulation of cell shape'
(\qval{6}) and in a variety of terms involving phosphate metabolism, such as
`nucleoside phosphate binding' (\qval{5}), `dephosphorylation' (\qval{3}) or
`phosphorylation' (\qval{2}), suggesting that this class may be enriched in
genes involved in signal transduction though the reason for this enrichment
remains unclear. The \bx{}-specific class did not show enrichment on any test,
consistent with our interpretation that this class is the result of random false
positive hits.

\subsection*{Predicted interactions of Mediator with Wnt and Ras pathways in
          \cel{}}
Previous work in \cel{}~\citep{Moghal2003,Zhang2000} has implicated \dpy{} as an
inhibitor of the Wnt and Ras pathways during the formation of the vulva and the
male tail. We obtained expression profiles for \gene{bar-1(ga80)} mutants as
well as loss-of-function and gain-of-function Ras mutants, \gene{let-60(n2021)}
and \gene{let-60(n1046gf)} respectively. We predicted that the \sy{}-specific
phenotypic class would exhibit the most significant overlap (assessed by a
hypergeometric enrichment test) with differentially expressed genes in
\gene{let-60(n1046gf)} mutants, whereas the \bx{}-associated phenotypic class
would exhibit the most significant overlap with \gene{bar-1(ga80)} mutants.

The \bx{}-specific class did not show a transcriptomic signature associated with
either the Wnt or the Ras pathway, consistent with our interpretation of this
class as false positive (Fig.~\ref{fig:wnt_stps}). All other classes showed
significant enrichment with genes perturbed in \gene{bar-1(ga80)}. Similarly,
\gene{let-60(n2021)} showed enrichment in all real phenotypic classes, with the
exception of the \emph{trans}-heterozygote specific class. Contrary to our
hypotheses, differentially expressed genes in \gene{let-60(n1046gf)} did not
show significant overlap with the \sy{}-specific phenotype, but they did show
significant overlap with all remaining real phenotypic classes.

\begin{figure}
  \includegraphics[width=0.5\textwidth]{alleleims/stp_pvals.pdf}
  \caption{
          \dpy{} phenotypic classes are statistically significantly enriched
          for signatures of \gene{let-60} (ras) and \gene{bar-1} (wnt)
          signaling.
          We tested whether the overlap between the differentially expressed
          genes in \gene{bar-1(ga80)}, \gene{let-60(n1046gf)} or
          \gene{let-60(n2021)} and the \dpy{} phenotypic classes was
          statistically significant using a hypergeometric enrichment test.
          Since the hypergeometric enrichment test is very sensitive to
          deviations from random, and since we suspect that there may be a broad
          genotoxic response to all mutants, we used a statistical significance
          threshold of $p < 10^{-10}$ (dashed black line).
  }
\label{fig:wnt_stps}
\end{figure}

\section*{Discussion}
\label{sec:conclusions}

\subsection{A conceptual framework for analyses of allelic series using
transcriptomic phenotypes}
Although transcriptomic phenotypes have been used for epistatic
analyses~\citep{Dixit2016,AngelesAlboresHIF,Angeles-Albores2017}, they have not
been used to study gene function in the context of an allelic series.
Outstanding challenges for transcriptomes in allelic series were how to count or
identify distinct phenotypes within the different transcriptomes, how to order
alleles in a severity hierarchy and how to order alleles in a dominance
hierarchy. In this work, we present solutions to these problems, and propose a
set of unifying concepts that we believe will be useful for future analyses. We
re-analyzed an allelic series of the Mediator subunit gene \dpy{} that had been
studied previously~\citep{Moghal2003}, recapitulating and extending previous
results as a proof of principle for our methodology. In our results, we derived
a set of methods that do not rely on the nature of the mutations. In the
subsequent discussion, we use the fact that the mutations we used were
truncations to derive further insights into the functional units present in this
gene.

To interpret our phenotypic classes in a biological context, we investigated
whether these phenotypic classes contained Ras and Wnt expression signatures.
Our attempts were partially successful, but a more rigorous analysis awaits the
availability of a larger mutant set to establish empirically the overlap that is
biologically significant. In part, we reason that some genes may form part of a
broad stress response. If that were the case, many mutants may share similar
transcriptomic signatures.

\subsection*{Phenotypic classes and their sequence requirements}
Because the mutations we used are truncations, our results suggest the existence
of various functional regions in \dpy{/MDT12} (Fig.~\ref{fig:domains}). These
functional regions could encode protein domains with biochemical activity, or
they could encode biochemically active amino acid motifs, such as nuclear
localization sequences or protein binding sites. These functional regions could
confer stability to the protein, thereby regulating its levels. As a caveat, we
note that we have interpreted the effects these mutations have in terms of their
putative effects at the protein level. In the case of our alleles, the relevant
homozygotes had wild-type \dpy{} mRNA levels, suggesting that these mutations
do not affect the stability of the mRNA.

The \sy{}-specific phenotypic class is likely controlled by a single functional
region, functional region 1 (FR1). Sequence necessary for wild-type FR1
functionality is encoded between amino acid positions 1 and 2,549, since this is
the sequence that is intact in the \emph{bx93} allele. We speculate that this
functional region may be the reason that \emph{bx93} is unable to complement the
Muv phenotype of \emph{sy622} in a sensitized \emph{let-23} background, since
\emph{trans}-heterozygotes in this background exhibit a semidominant Muv
phenotype.  The \sy{}-associated phenotypic class is likely controlled by a
second functional region, functional region 2 (FR2), and some necessary
sequences for wild-type function are encoded between amino acid positions 1,698
and 2,549, but additional sequence could lie between amino acids 1 and 1,698. It
is unlikely that FR1 and FR2 are identical because their dominance behaviors are
very different. The \bx{} allele was largely dominant over the \sy{} allele for
the \bx{}-associated class, but gene expression in this class was perturbed in
both homozygotes. The perturbations were greater for \sy{} homozygotes than for
\bx{} homozygotes. This behavior can be explained if the \bx{}-associated class
is controlled jointly by two distinct effectors, functional regions 3 and 4
(FR3, FR4, see Fig.~\ref{fig:domains}). Such a model would propose that the
sequences necessary for FR3 functionality are within the interval 1 and 2,549,
and some sequences necessary for FR4 functionality are encoded between positions
2549 and 3499. This model explains how expression levels of the
\emph{bx93}-associated phenotypic class in the \emph{trans}-heterozygote are
complemented to the levels of the \emph{bx93} homozygote, because FR3 is
complemented in \emph{trans}, but FR4 is defective. Thus, FR3 encodes a
functionality that is not dosage-dependent. One possibility is that FR3 is
equivalent to FR1 or FR2, and FR4 modifies activity of either of these regions
at a subset of loci. A rigorous examination of this model will require studying
many alleles that mutate the region between Q1689 and Q2549 using homozygotes
and \emph{trans}-heterozygotes.

\begin{figure}
  \centering{}
  \includegraphics[width=0.5\textwidth]{alleleims/inferred_domains.pdf}
  \caption{
          The functional regions associated with each phenotypic class can be
          mapped intragenically. The number of genes associated with each class
          is shown. The \bx{}-associated class may be controlled by two
          functional regions. FR1 is a dosage-sensitive unit. FR2 and FR3 could
          be redundant if FR4 is a modifier of FR2 functionality at
          \bx{}-associated loci. Note that the \bx{}-associated phenotypic class
          is actually three classes merged together. Two of these classes are DE
          in \bx{} homozygotes and one other genotype. Our analyses suggested
          that these two classes are likely the result of false negative hits
          and genes in these classes should be differentially expressed in all
          three genotypes, so we merged these three classes together
          (see~\hyperref[sec:methods]{Methods}). }
\label{fig:domains}
\end{figure}

We also found a class of transcripts that had perturbed levels in
\emph{trans}-heterozygotes only; its biological significance is unclear.
Phenotypes unique to \emph{trans}-heterozygotes are often the result of physical
interactions such as homodimerization, or dosage reduction of a toxic
product~\citep{Yook2005}. In the case of \dpy{/MDT12} orthologs, these
explanations seem unlikely since \protein{dpy-22} is a monomeric subunit of the
CKM.\@ Another possibility is that the \emph{trans}-heterozygote-specific class
is the result of complex tissue cross-talk. Massive single-cell RNA-seq of
\cel{} has recently been reported~\citep{Cao2017}, and this tool could provide
valuable information regarding this hypothesis. Another possibility is that the
\emph{cis}-marker we used for the \emph{bx93} allele, \gene{dpy-6(e14)}, which
we assumed to be recessive in all phentoypes, actually has dominant
transcriptomic phenotype.

\subsection*{Occam's razor}
Transcriptomic phenotypes generate large amounts of differential gene expression
data, so false positive and false negative rates can lead to spurious phenotypic
classes whose putative biological significance is misleading. Such
artifacts are particularly likely when a phenotypic class is small. Notably,
errors of interpretation cannot be avoided by setting a more stringent $q$-value
cut-off: doing so will decrease the false positive rate, but increase the false
negative rate, which will in turn produce smaller phenotypic classes than
expected. Our method tries to avoid this pitfall by using total error rate
estimates to assess the plausibility of each class, though a major drawback is
that it relies on a subjective estimation of the false negative rate. These
conclusions are of broad significance to research where highly multiplexed
measurements are compared to identify similarities and differences in the
genome-wide behavior of a single variable under multiple conditions.

We have shown that transcriptomes can be used to study allelic series in the
context of a large, pleiotropic gene. We identified separable phenotypic classes
that would otherwise be obscured by other methods, correlated each class to a
functional region, and identified sequence requirements for each region. Given
the importance of allelic series for characterizing gene function and their
roles in specific genetic pathways, we are optimistic that this method will be a
useful addition to the geneticist's arsenal.

  \printbibliography[heading=subbibliography]
\end{refsection}

\chapter{Tissue enrichment analysis for \cel{} genomics}
\begin{refsection}
  \input{chapters/tea.tex}
  \printbibliography[heading=subbibliography]
\end{refsection}

\chapter{Two new functions in the WormBase Enrichment Suite}
\begin{refsection}
  \section{Description}
Genome-wide experiments routinely generate large amounts of data that can be
hard to interpret biologically. A common approach to interpreting these results
is to employ enrichment analyses of controlled languages, known as ontologies,
that describe various biological parameters such as gene molecular or biological
function. In \cel{}, three distinct ontologies, the Gene Ontology (GO),
Anatomy Ontology (AO), and the Worm Phenotype Ontology (WPO) are used to
annotate gene function, expression and phenotype,
respectively~\citep{TheGeneOntologyConsortium2000a,Lee2003,Schindelman2011}.
Previously, we developed software to test datasets for enrichment of anatomical
terms, called the Tissue Enrichment Analysis (TEA)
tool~\citep{Angeles-Albores2016}. Using the same hypergeometric statistical
method, we extend enrichment testing to include WPO and GO, offering a unified
approach to enrichment testing in \cel{}. The WormBase Enrichment Suite can
be accessed via a user-friendly interface at
\url{http://www.wormbase.org/tools/enrichment/tea/tea.cgi}.

To validate the tools, we analyzed a previously published extracellular vesicle
(EV)-releasing neuron (EVN) signature gene set derived from dissociated ciliated
EV neurons\citep{Wang2015} using the WormBase Enrichment Suite based on the
WS262 WormBase release. TEA correctly identified the CEM, hook sensillum and IL2
neuron as enriched tissues. The top phenotype associated with the EVN signature
was chemosensory behavior. Gene Ontology enrichment analysis showed that cell
projection and cell body were the most enriched cellular components in this gene
set, followed by the biological processes neuropeptide signaling pathway and
vesicle localization further down. The tutorial script used to generate the
figure above can be viewed at:
\url{https://github.com/dangeles/TissueEnrichmentAnalysis/blob/master/tutorial/Tutorial.ipynb}

The addition of Gene Enrichment Analysis (GEA) and Phenotype Enrichment Analysis
(PEA) to WormBase marks an important step towards a unified set of analyses that
can help researchers to understand genomic datasets. These enrichment analyses
will allow the community to fully benefit from the data curation ongoing at
WormBase.


\section*{Methods}
Using the methods described in \citet{Angeles-Albores2016}, we
generated ontology dictionaries using the Anatomy, Phenotype and Gene Ontology
annotations for \cel{}. The dictionary similarity parameter was set to 95%
for all ontologies. The annotation per term minimum was set to 33 annotations
for the AO, a 50 annotations for the WPO, and 33 annotations for GO. Terms
within the dictionary are tested using a hypergeometric probability test and
corrected using the Benjamini-Hochberg step-up algorithm. In WS262, there are
1320 anatomy terms, 1117 phenotypes, and 3025 GO terms that have at least 11
genes annotated to them. The dictionaries are freely accessible using the Python
version of the Suite, which can be installed using the pip tool for Python
libraries: \texttt{pip install tissue\_enrichment\_analysis}. The dictionary can
then be automatically downloaded by importing the enrichment analysis library in
a Python script by writing \texttt{import tissue\_enrichment\_analysis as ea}.
The dictionaries can then be downloaded by  typing
\texttt{ea.fetch\_dictionary(dict)} into Python, where `dict ` is one of the
strings `tissue', `phenotype' or `go' to specify which dictionary to download.
If the function does not receive an argument, the dictionary corresponding to
the AO is downloaded by default. See the tutorial above for an example
implementation.

  \printbibliography[heading=subbibliography]
\end{refsection}


\chapter*{Conclusion}
\addcontentsline{toc}{chapter}{Conclusion}

I have tried to demonstrate that transcriptomes are phenotypes, and I have tried
to show examples of how these phenotypes can be rigorously manipulated. The work
is not without flaws, and some of it may even be wrong. However, I believe the
principles of the work are correct: Batesonian epistasis exists in
transcriptomes and should be a quantity of immense interest to us because it
does not require a null hypothesis; allelic series must be explored using
transcriptome-wide dominance; and we must continue to develop tools that make
use of the enormous amount of information that the scientific community is
actively generating. It is my hope that these concepts prove useful to the
greater scientific community.

% \printbibliography[heading=bibintoc]

% \appendix

\printindex

% \theendnotes{}

\end{document}
