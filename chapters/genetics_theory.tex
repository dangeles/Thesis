
\newcommand{\Freg}{\( F_{reg} \)}
\newcommand{\pder}[3]{\frac{\partial^{#3} {#1}}
                           {\partial {#2}^{#3}}
                     } % ith partial derivative with respect to something
\newcommand{\tayX}[2]{\frac{#1^{#2}}{#2!}} % x^i/i!


\section*{Abstract}
\textbf{Genetics is a powerful method that can be used to probe the molecular
function of individual genes and to reconstruct genetic interaction networks. A
cornerstone of genetics is the concept of epistasis: The ability of the
phenotype associated with a null allele of one gene to block the phenotype of a
null allele associated with a second gene. Epistasis is a widespread phenomenon
in biology, and it has been observed using phenotypes that span many orders of
magnitude in length-scales. In spite of its importance, a theoretical derivation
for why epistasis occurs at so many scales is lacking. Here, we use statistical
mechanics to derive epistasis from first principles.
}

\section*{Introduction}
Imagine, if you will, a partition function of
unbounded complexity. Imagine that there are thousands of potential variables
that could, but do not have to, participate in the system. Further imagine that
this system is not immediately experimentally tractable: The number of particles
in the system cannot be easily controlled, the various energies, enthalpies and
entropies of the system cannot be measured, and an analytical function cannot be
satisfactorily written from theoretical principles. Such a problem might at
first sight appear to be intractable through the methods of statistical
mechanics, or may require highly complex numerical methods to estimate
the properties of the partition function.

This thermodynamic regime is the regime occupied by many biological systems.
Bacteria, archaea, fungi or animals consist of a large and unknown number of
particles (proteins) encoded within genes in a genome. Any one characteristic
(phenotype) in an organism is the result of a large and unknown number of
particles interacting in a large and unknown number of configurations.

For systems where little or nothing is known about the factors controlling a
phenotype, biology relies on a powerful empirical method: genetics.
Briefly, in classical genetics, mutants of random genes are generated until a
gene is found that, when removed, causes a mutant phenotype. Mutants that
exhibit the desired mutant phenotypes can be combined to generate double mutants
and the phenotype of the resulting double mutant is inspected. If the two genes
under investigation exhibit a phenomenon known as classical (or Batesonian)
epistasis, then these two genes are said to have a genetic interaction, and they
can be ordered into a genetic pathway where one gene activates or inhibits the
second. Classical epistasis is not equivalent to the concept of epistasis used
in population genetics or human genetics~\citep{Cordell2002}. In these fields,
epistasis is represented by second order interaction terms between arbitrary
mutations in a linear or log-linear statistical model. There has been
significant work on the effect these second order terms have on these
statistical models~\citep{Crow2010,Mackay2014}. However, these statistical
models are not grounded in a principled theory of genetic interaction and the
presence or absence of epistasis is dependent on the choice of statistical
model~\citep{Cordell2002}. In this text, we restrict our studies to classical
epistasis.

Genetics has been of major importance for finding the genes that control
phenotypes of interest and for ordering them into networks that are amenable for
biochemical or biophysical characterization. The choice of phenotype is
important, and the introduction of new phenotypes has led to significant
breakthroughs. The phenotypes used for genetic analysis include animal
morphology~\citep{Brenner1974} and development~\citep{Jurgens1984},
behavior~\citep{Benzer1967}, cellular differentiation,
metabolism~\citep{Beadle499}, and most recently gene expression
levels~\citep{Angeles-Albores2018a,Hughes2000,Capaldi2008}. The ability of
genetics to establish interactions between particles using phenotypes that vary
by 6--8 orders of magnitude of length scales has been extremely useful and is
deeply intriguing from a theoretical perspective. In spite of its strong logical
foundation and enormous empirical evidence of its usefulness, there is no
general theoretical description for why genetics is so effective.

\section*{Statistical Mechanics of Genetic Interactions}
We will derive epistasis in gene expression phenotypes using a toy model for
gene expression~\citep{Garcia2007,Bintu2005a}. This model can be derived from
the assumption that the level at which a gene is expressed is directly
proportional to the probability that RNA Polymerase II is bound at that gene's
promoter. This probability depends upon the RNA Polymerase levels, \( \rho \),
and on other factors, \( \{A, B, \ldots \} \) that can bind the promoter and RNA
polymerase:

\begin{equation}
  p_{bound}(A, B, \ldots) = \frac{1}{1 + \frac{1}{\rho F_{reg}(A, B, \ldots)}}.
  \label{eq:pbound}
\end{equation}

\(p_{bound}(A, B, \ldots)\) is the probability that RNA Polymerase is bound at
the locus of interest. \Freg{} is a rational function,

\begin{equation*}
    F_{reg}(A, B) = P(A, B)/Q(A, B).
\end{equation*}
It represents the effective number of RNA polymerases at the promoter of
interest and in general cannot be analytically determined for all but the
simplest systems. To ensure that Eq.~\ref{eq:pbound} is a probability, the range
of \Freg{} is restricted to positive real numbers. This factor is used to model
a variety of transcription factors, such as activators, or inhibitors, and the
physical interactions between them. \(A\) and \(B\) represent the activities of
the gene products of genes \gene{a}, \gene{b}. The variables \(X\) are related
to the physical number of proteins of X, \(X_{\text{protein}}\) through the
equation:

\begin{equation*}
    X = \frac{X_{\text{protein}}e^{-\varepsilon_{xd}}}{N_{NS}}.
\end{equation*}
Here, \(\varepsilon_{xd}\) refers to the energy of binding of protein X to
DNA at the promoter (in units of \(k_B T\)), and \(N_{NS}\) refers to the total
number of non-specific sites on the genome. A major assumption in deriving this
equation is that all proteins are bound to either the promoter or alternatively
to non-specific sites in the genome. A further assumption is that \(N_{NS} \gg
X\).

We are interested in what occurs when either variable \(A\), \(B\) or both
\(A\) and \(B\) are set to zero instead of the levels found in a non-mutant, or
wild-type, reference organism (\(X_{wt}\)). Specifically, we will explore the
constraints on the functional form between \(A\) and \(B\) such that the
distribution of gene expression levels in a mutant lacking protein A is
completely independent of the levels of protein B, even though in the general
case of all non-zero levels of protein A, the probability of RNAP binding is
conditional on the levels of both A and B.

Experimentally, this is tested by generating single and double mutants and
measuring the expression level of a reporter gene. We search for gene pairs,
\gene{a} and \gene{b}, where the expression level of a reporter in a mutant
lacking protein A is equal to the expression level of the reporter in a mutant
lacking both proteins,

\begin{equation}
    p_{bound}(A=0, B=0) = p_{bound}(A=0, B_{wt}).
    \label{eq:epistasis}
\end{equation}

This condition is trivially satisfied if \(B\) does not play a role in
controlling the gene expression level of the reporter, so we will only consider
cases where perturbing the value of \(B\) away from \(B_{wt}\) changes the
expression levels of the reporter gene.

The identity in Eq.~\ref{eq:epistasis} is called classical (or Batesonian)
epistasis, and demonstrates that a null allele of one gene (\gene{a}) can mask
the phenotype associated with a null allele of a second gene (\gene{b}). The
gene that is masked is said to be \emph{hypostatic}, while the masking gene is
\emph{epistatic}. When a pair of genes shows epistasis, there is a genetic
interaction between gene \gene{a} and gene \gene{b} (note: the genetic
interaction is not said to occur between the two proteins). Since \(p_{bound}\)
depends on proteins A and B only through \(F_{reg}(\cdot,\cdot)\),
Eq.~\ref{eq:epistasis} can be re-written in terms of the regulatory function.

\begin{equation*}
    F_{reg}(A=0, B=0) = F_{reg}(A=0, B_{wt}).
\end{equation*}
We can approximate the right hand side of this equation as a Taylor function of
\(B\) around 0, letting \(F_{reg}(A=0, B=0) = \phi \):

\begin{equation*}
  \phi = \phi + \sum_i \pder{}{B}{i} F(A=0, B=0) \frac{B_{wt}^i}{i!}.
  % \label{eq:TaylorF}
\end{equation*}
This equation can be satisfied for arbitrary values of \(B\) if and only if:

\begin{equation}
  \pder{}{B}{i} F(A=0, B=0) = 0.
  \label{eq:epiT}
\end{equation}
Thus, enforcing epistasis (Eq.~\ref{eq:epistasis}) is equivalent to constraining
all partial derivates with respect to the \emph{hypostatic} variable, \(B\), of
\(F_{reg}\) at \((0, 0)\) to sum to zero. Since \Freg{} is a rational function
we can re-write Eq.~\ref{eq:epiT} using the chain rule,

\begin{equation*}
  \sum_{i=1}^\infty
    \Bigg {[} \pder{F_{reg}}{P}{i}  % first partial
              \pder{P}{B}{i}  % second partial
    + \pder{F_{reg}}{Q}{i}  % first partial
      \pder{Q}{B}{i}  % second partial
    \Bigg {]}_{A=0, B=0}
    = 0,
  \label{eq:key}
\end{equation*}
which reveals that all partial derivatives of \(P\) and \(Q\) with respect to
\(B\) must be zero at the point \((A=0, B=0)\).

We cast the polynomials \(P\) and \(Q\) into the general form

\begin{equation*}
    X = \sum_{j, k=0}^\infty \lambda^X_{jk} A^j B^k.
\end{equation*}
Using this general form, we will now find the constraints on \(\lambda^X_{jk}\)
such that all the partial derivatives of this polynomial family vanish when
\(A=0\). From inspection, all terms of order \(j \geq 1\) will be zero when
\(A =0\). Then, it follows that if \(\lambda^X_{0,k} = 0, \forall k>0\), all the
partial derivates of \(P\) and \(Q\) with respect to \(B\) are 0 when \(A=0\).
Thus, if two genes, \gene{a} and \gene{b} satisfy Eq.~\ref{eq:epistasis} when
they are mutated, then \(P\) and \(Q\) can be written as:

\begin{equation}
  X = \lambda^X_{00} + \sum_{j = 1}^\infty \lambda^X_{j0} A^j +
                       \sum_{j, k \geq 1}^\infty \lambda^X_{jk} A^j B^k.
\label{eq:mesostates}
\end{equation}

Where \(X\) is either \(P\) or \(Q\). We refer to each term in
Eq.~\ref{eq:mesostates} as a \emph{mesostate}. Briefly, a mesostate is a
combination of microstates containing a defined set of species with an unknown
stoichiometric distribution. In Eq.~\ref{eq:mesostates} there are three
mesostates. If a mesostate is compatible with a set of epistatic relationships,
then it must be non-empty in either \(P\) or \(Q\).

\subsection{Epistasis is transitive}
Suppose there are three genes, \gene{a}, \gene{b} and \gene{c} encoding proteins
A, B and C respectively. Suppose further that epistasis analyses performed using
a specific reporter as an expression phenotype show that \gene{a} is epistatic
over \gene{b} and \gene{b} is epistatic over \gene{c}. Is \gene{a} epistatic
over \gene{c}?

Once again, we let \Freg{} be a rational function of the polynomials \(P\) and
\(Q\). The general form of these polynomials is:

\begin{equation*}
    X = \sum_{j, k, l} \lambda^X_{jkl} A^j B^k C^l
\end{equation*}

Since \gene{a} is epistatic over \gene{b}, it follows that
\(\lambda^X_{0, k, l} = 0\forall k > 0\). Since \gene{b} is epistatic over \gene{c},
it follows that \(\lambda^X_{j, 0, l} = 0 \forall l>0\). Therefore, these
polynomials can be written as

\begin{multline}
  X =
  \lambda^X_{000}
  +
  \sum_{j=1}^\infty \lambda^X_{j00} A^j
  +
  \sum_{j, k \geq 1}^\infty \lambda^X_{jk0} A^j B^k\\
  +
  \sum_{j, k, l \geq 1}^\infty \lambda^X_{jkl} A^j B^k C^l.
\end{multline}
From this functional form, it is clear that \gene{a} is epistatic over \gene{c}.
Therefore, epistasis is transitive.

\subsection{Epistasis is hierarchical}
Suppose there are three genes, \gene{a}, \gene{b} and \gene{d} encoding proteins
A, B and D respectively. Suppose further that epistasis analyses performed using
a specific reporter as an expression phenotype show that \gene{a} is epistatic
over \gene{b} and \gene{d} is epistatic over \gene{b}. Must it be the case that
either \gene{a} is epistatic over \gene{d}, or \gene{d} is epistatic over
\gene{a}?

Once again, we let \Freg{} be a rational function of the polynomials \(P\) and
\(Q\). The general form of these polynomials is:

\begin{equation*}
    X = \sum_{j, k, l} \lambda^X_{jkl} A^j B^k D^l
\end{equation*}

Since \gene{a} is epistatic over \gene{b}, it follows that
\(\lambda^X_{0, k, l} = 0,\forall k > 0\). Since \gene{d} is epistatic over \gene{b},
it follows that \(\lambda^X_{j, k, 0} = 0, \forall l>0\). Therefore, these
polynomials can be written as

\begin{multline}
  X =
  \lambda^X_{000}
  +
  \sum_{j = 1}^\infty \lambda^X_{j00} A^j
  +
  \sum_{j, k \geq 1}^\infty \lambda^X_{jk0} A^j B^k\\
  +
  \sum_{l = 1}^\infty \lambda^X_{00l} D^l
  +
  \sum_{k, l \geq 1}^\infty \lambda^X_{0kl} B^k D^l\\
  +
  \sum_{j, k, l \geq 1}^\infty \lambda^X_{jkl} A^j B^k D^l.
\label{eq:abd}
\end{multline}
This functional form means that we cannot conclude anything about the epistatic
relationship between \gene{a} and \gene{d} without generating a double mutant of
\gene{a} and \gene{d}.

The results from the preceding two sections mean that epistasis is similar to
an inequality statement. Therefore, we propose the notation:
\begin{equation}
  a > b
\end{equation}
to represent the genetic epistasis of gene \gene{a} over \gene{b} as defined by
Eq.~\ref{eq:epistasis}.

\subsection{Epistasis enables qualitative functional inferences}

Suppose that two genes, \gene{a} and \gene{b}, \(a > b\). Further suppose that
the phenotypes can be arranged in the following order:

\begin{equation*}
    p_{wt}(A_{wt}, B=0) < p_{wt}(A=0, B_{wt}) = p_{wt}(A_{wt}, B_{wt}).
\end{equation*}
This order can be immediately rephrased in terms of \Freg{},
\begin{equation*}
    F_{reg}(A_{wt}, B=0) < F_{reg}(A=0, B_{wt}) = F_{reg}(A_{wt}, B_{wt}).
\end{equation*}

Since \Freg{} is a function of polynomials \(P\) and \(Q\), both of which have
the functional form,

\begin{equation*}
    X = \lambda^X_{00} + \sum_{j=0}^\infty \lambda^X_{j0}A^j +
          \sum_{j,k \geq 1}^\infty \lambda^X_{jk}A^j B^k
\end{equation*}

Since we know that at least one term in each mesostate of \(P\) and \(Q\) must
be non-zero, we conclude that the effective activity of A must be 0. With this
information, the family of functions \(P\) and \(Q\) that will satisfy this
hierarchy is:

\begin{equation}
  Q(A_{wt}, B_{wt}) = \lambda^Q_{00} + \sum_j \lambda^Q_{j0}A^j +
                        \sum_{j,k\geq 1} \lambda^Q_{jk}A_{wt}^j B_{wt}^k
\end{equation}
and

\begin{equation}
  P(A_{wt}, B_{wt}) = \lambda^P_{00} +
                        \sum_{j,k \geq 1} \lambda^P_{jk}A_{wt}^j B_{wt}^k.
\end{equation}
For these arguments to be true, it must also be the case that
\(B_{wt} \gg A_{wt}\) (equality is only achieved in the case when either
\(B_{wt}\) becomes infinite or \(A_{wt}\) is zero). Genetically, gene \gene{b}
is a net inhibitor of gene \gene{a}, and gene \gene{a} is a net genetic
inhibitor of our reporter phenotype.

We consider a different epistatic relationship between two different genes,
\gene{c} and \gene{d}, such that \(c > d\) and the phenotypes can be ordered:

\begin{equation*}
    F_{reg}(C_{wt}, D_{wt}) < F_{reg}(C_{wt}, D=0) < F_{reg}(C=0, D_{wt}).
\end{equation*}
We recall that:

\begin{equation*}
    F_{reg}(C=0, D_{wt}) = \frac{\lambda_{00}^P}{\lambda^Q_{00}}
\end{equation*}

A suitable family of functions for this hierarchy is:

\begin{equation}
  Q(C_{wt}, D_{wt}) = \lambda^Q_{00} + \sum_j \lambda^Q_{j0}C^j +
                        \sum_{j,k\geq 1} \lambda^Q_{jk}C_{wt}^j D_{wt}^k
\end{equation}
and

\begin{equation}
  P(A_{wt}, B_{wt}) = \lambda^P_{00} +
                        \sum_{j,k \geq 1} \lambda^P_{jk}C_{wt}^j D_{wt}^k,
\end{equation}

subject to the constraint:
\begin{equation}
  0 < \sum_{j,k\geq 1} (\lambda^Q_{jk} - \lambda_{jk}^P)C_{wt}^j D_{wt}^k.
\end{equation}

This family of functions allows us to conclude that gene \gene{c} is a net
genetic inhibitor of our reporter phenotype, and \gene{d} is a net promoter of
the genetic activity of gene \gene{c}.

\subsection{Interpretation of epistasis for statistical mechanical systems}
We have shown that classical epistasis is an identity between single and double
mutants (Eq.~\ref{eq:epistasis}) that is the result of nested polynomials. To
understand epistasis, we introduce the concept of \emph{mesostates}. We define
mesostates as combinations of microstates involving a defined set of species
with an unknown stoichiometric distribution. In Eq.~\ref{eq:abd}, for example,
is a sum of six mesostates. The first mesostate represents promoter leakiness;
the second consists of all microstates that depend on the presence of protein A
to form; the third consists of all microstates that depend on the presence of
proteins A and B to form; the fourth consists of all microstates that depend on
the presence of protein D to form; the fifth consists of all microstates that
depend on the presence of protein B and D to form; and the sixth consists of all
microstates that depend on the presence of proteins A, B and D to form.
Epistasis on its own cannot tell us how many microstates correspond to a single
mesostate, but it can tell us what mesostates are not accessible to the system,
thus ruling out families of microstates.

Throughout this text, we have assumed that our imaginary proteins participated
directly in binding to the promoter of interest. However, we can dispense with
this requirement, and instead imagine that these proteins function as switches
that permit the existence of specific mesostates accessible to the promoter.
This is the reason why we refer to epistatic interactions as genetic
interactions: Epistasis does not provide any guarantee that the gene products
ever interact physically, chemically or even that they coexist in the same space
at the same time.

Here, we have shown that classical genetics, which has had a rich history over
the past century and is a cornerstone of modern biology, is equivalent to a
perturbative, parameter-free study of the partition function of a thermodynamic
system. Though we have limited ourselves to applying this method to gene
expression phenotypes, this approach is generalizable, and in fact, is not even
limited to biological systems. We believe that the statistical mechanical basis
for genetics explains its ability to explain phenotypes that span many orders of
magnitude in time, space and molecular complexity.
