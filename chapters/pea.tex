\section{Description}
Genome-wide experiments routinely generate large amounts of data that can be
hard to interpret biologically. A common approach to interpreting these results
is to employ enrichment analyses of controlled languages, known as ontologies,
that describe various biological parameters such as gene molecular or biological
function. In \cel{}, three distinct ontologies, the Gene Ontology (GO),
Anatomy Ontology (AO), and the Worm Phenotype Ontology (WPO) are used to
annotate gene function, expression and phenotype,
respectively~\citep{TheGeneOntologyConsortium2000a,Lee2003,Schindelman2011}.
Previously, we developed software to test datasets for enrichment of anatomical
terms, called the Tissue Enrichment Analysis (TEA)
tool~\citep{Angeles-Albores2016}. Using the same hypergeometric statistical
method, we extend enrichment testing to include WPO and GO, offering a unified
approach to enrichment testing in \cel{}. The WormBase Enrichment Suite can
be accessed via a user-friendly interface at
\url{http://www.wormbase.org/tools/enrichment/tea/tea.cgi}.

To validate the tools, we analyzed a previously published extracellular vesicle
(EV)-releasing neuron (EVN) signature gene set derived from dissociated ciliated
EV neurons\citep{Wang2015} using the WormBase Enrichment Suite based on the
WS262 WormBase release. TEA correctly identified the CEM, hook sensillum and IL2
neuron as enriched tissues. The top phenotype associated with the EVN signature
was chemosensory behavior. Gene Ontology enrichment analysis showed that cell
projection and cell body were the most enriched cellular components in this gene
set, followed by the biological processes neuropeptide signaling pathway and
vesicle localization further down. The tutorial script used to generate the
figure above can be viewed at:
\url{https://github.com/dangeles/TissueEnrichmentAnalysis/blob/master/tutorial/Tutorial.ipynb}

The addition of Gene Enrichment Analysis (GEA) and Phenotype Enrichment Analysis
(PEA) to WormBase marks an important step towards a unified set of analyses that
can help researchers to understand genomic datasets. These enrichment analyses
will allow the community to fully benefit from the data curation ongoing at
WormBase.


\section*{Methods}
Using the methods described in \citet{Angeles-Albores2016}, we
generated ontology dictionaries using the Anatomy, Phenotype and Gene Ontology
annotations for \cel{}. The dictionary similarity parameter was set to 95%
for all ontologies. The annotation per term minimum was set to 33 annotations
for the AO, a 50 annotations for the WPO, and 33 annotations for GO. Terms
within the dictionary are tested using a hypergeometric probability test and
corrected using the Benjamini-Hochberg step-up algorithm. In WS262, there are
1320 anatomy terms, 1117 phenotypes, and 3025 GO terms that have at least 11
genes annotated to them. The dictionaries are freely accessible using the Python
version of the Suite, which can be installed using the pip tool for Python
libraries: \texttt{pip install tissue\_enrichment\_analysis}. The dictionary can
then be automatically downloaded by importing the enrichment analysis library in
a Python script by writing \texttt{import tissue\_enrichment\_analysis as ea}.
The dictionaries can then be downloaded by  typing
\texttt{ea.fetch\_dictionary(dict)} into Python, where `dict ` is one of the
strings `tissue', `phenotype' or `go' to specify which dictionary to download.
If the function does not receive an argument, the dictionary corresponding to
the AO is downloaded by default. See the tutorial above for an example
implementation.
